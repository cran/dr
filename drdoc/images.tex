\batchmode
\documentclass{article}
\RequirePackage{ifthen}


\usepackage[ae]{Rd}
\usepackage{html}%
\providecommand{\Splus}{{\normalfont\textsf{Splus}}{}}%
\providecommand{\arc}{{\em Arc}}%
\providecommand{\xls}{{\em Xlisp-Stat}}%
\providecommand{\dcode}[1]{{\small{\tt #1}}}%
\providecommand{\ir}{{\tt dr}}%
\providecommand{\dr}{{\tt dr}}%
\providecommand{\sir}{{\sffamily sir}}%
\providecommand{\save}{{\sffamily save}}%
\providecommand{\phd}{{\sffamily phd}}%
\providecommand{\phdy}{{\sffamily phdy}}%
\providecommand{\phdq}{{\sffamily phdq}}%
\providecommand{\phdres}{{\sffamily phdres}}
%
\renewcommand{\span}{{\mathcal S}}%
\providecommand{\E}{\mbox{E}}%
\providecommand{\N}{\mbox{N}}

%
\newenvironment{references}{%%
  \csname section\endcsname{References}%%
    \begin{list}%%
            {}%%
            {\setlength{\leftmargin}{.25in}\setlength{\itemindent}{-.25in}}}%%
   {\end{list}}%
\providecommand{\jasa}{{\it Journal of the American Statistical Association}}%
\providecommand{\jrssa}{{\it Journal of the Royal Statistical Society, Series A}}%
\providecommand{\jrssb}{{\it Journal of the Royal Statistical Society, Series B}}%
\providecommand{\jrssc}{{\it Applied Statistics}}%
\providecommand{\techx}{{\it Technometrics}}%
\providecommand{\Wiley}{New York: Wiley}


\title{Dimension Reduction Regression in \R}
\author{Sanford Weisberg\\
{\small \sl
School of Statistics, University of Minnesota, St. Paul,
MN 55108-6042.}\\
{\small Supported by National Science Foundation Grant DUE 0109756.}}
\date{\today}

\usepackage[dvips]{color}


\pagecolor[gray]{.7}

\usepackage[]{inputenc}



\makeatletter

\makeatletter
\count@=\the\catcode`\_ \catcode`\_=8 
\newenvironment{tex2html_wrap}{}{}%
\catcode`\<=12\catcode`\_=\count@
\newcommand{\providedcommand}[1]{\expandafter\providecommand\csname #1\endcsname}%
\newcommand{\renewedcommand}[1]{\expandafter\providecommand\csname #1\endcsname{}%
  \expandafter\renewcommand\csname #1\endcsname}%
\newcommand{\newedenvironment}[1]{\newenvironment{#1}{}{}\renewenvironment{#1}}%
\let\newedcommand\renewedcommand
\let\renewedenvironment\newedenvironment
\makeatother
\let\mathon=$
\let\mathoff=$
\ifx\AtBeginDocument\undefined \newcommand{\AtBeginDocument}[1]{}\fi
\newbox\sizebox
\setlength{\hoffset}{0pt}\setlength{\voffset}{0pt}
\addtolength{\textheight}{\footskip}\setlength{\footskip}{0pt}
\addtolength{\textheight}{\topmargin}\setlength{\topmargin}{0pt}
\addtolength{\textheight}{\headheight}\setlength{\headheight}{0pt}
\addtolength{\textheight}{\headsep}\setlength{\headsep}{0pt}
\setlength{\textwidth}{349pt}
\newwrite\lthtmlwrite
\makeatletter
\let\realnormalsize=\normalsize
\global\topskip=2sp
\def\preveqno{}\let\real@float=\@float \let\realend@float=\end@float
\def\@float{\let\@savefreelist\@freelist\real@float}
\def\liih@math{\ifmmode$\else\bad@math\fi}
\def\end@float{\realend@float\global\let\@freelist\@savefreelist}
\let\real@dbflt=\@dbflt \let\end@dblfloat=\end@float
\let\@largefloatcheck=\relax
\let\if@boxedmulticols=\iftrue
\def\@dbflt{\let\@savefreelist\@freelist\real@dbflt}
\def\adjustnormalsize{\def\normalsize{\mathsurround=0pt \realnormalsize
 \parindent=0pt\abovedisplayskip=0pt\belowdisplayskip=0pt}%
 \def\phantompar{\csname par\endcsname}\normalsize}%
\def\lthtmltypeout#1{{\let\protect\string \immediate\write\lthtmlwrite{#1}}}%
\newcommand\lthtmlhboxmathA{\adjustnormalsize\setbox\sizebox=\hbox\bgroup\kern.05em }%
\newcommand\lthtmlhboxmathB{\adjustnormalsize\setbox\sizebox=\hbox to\hsize\bgroup\hfill }%
\newcommand\lthtmlvboxmathA{\adjustnormalsize\setbox\sizebox=\vbox\bgroup %
 \let\ifinner=\iffalse \let\)\liih@math }%
\newcommand\lthtmlboxmathZ{\@next\next\@currlist{}{\def\next{\voidb@x}}%
 \expandafter\box\next\egroup}%
\newcommand\lthtmlmathtype[1]{\gdef\lthtmlmathenv{#1}}%
\newcommand\lthtmllogmath{\lthtmltypeout{l2hSize %
:\lthtmlmathenv:\the\ht\sizebox::\the\dp\sizebox::\the\wd\sizebox.\preveqno}}%
\newcommand\lthtmlfigureA[1]{\let\@savefreelist\@freelist
       \lthtmlmathtype{#1}\lthtmlvboxmathA}%
\newcommand\lthtmlpictureA{\bgroup\catcode`\_=8 \lthtmlpictureB}%
\newcommand\lthtmlpictureB[1]{\lthtmlmathtype{#1}\egroup
       \let\@savefreelist\@freelist \lthtmlhboxmathB}%
\newcommand\lthtmlpictureZ[1]{\hfill\lthtmlfigureZ}%
\newcommand\lthtmlfigureZ{\lthtmlboxmathZ\lthtmllogmath\copy\sizebox
       \global\let\@freelist\@savefreelist}%
\newcommand\lthtmldisplayA{\bgroup\catcode`\_=8 \lthtmldisplayAi}%
\newcommand\lthtmldisplayAi[1]{\lthtmlmathtype{#1}\egroup\lthtmlvboxmathA}%
\newcommand\lthtmldisplayB[1]{\edef\preveqno{(\theequation)}%
  \lthtmldisplayA{#1}\let\@eqnnum\relax}%
\newcommand\lthtmldisplayZ{\lthtmlboxmathZ\lthtmllogmath\lthtmlsetmath}%
\newcommand\lthtmlinlinemathA{\bgroup\catcode`\_=8 \lthtmlinlinemathB}
\newcommand\lthtmlinlinemathB[1]{\lthtmlmathtype{#1}\egroup\lthtmlhboxmathA
  \vrule height1.5ex width0pt }%
\newcommand\lthtmlinlineA{\bgroup\catcode`\_=8 \lthtmlinlineB}%
\newcommand\lthtmlinlineB[1]{\lthtmlmathtype{#1}\egroup\lthtmlhboxmathA}%
\newcommand\lthtmlinlineZ{\egroup\expandafter\ifdim\dp\sizebox>0pt %
  \expandafter\centerinlinemath\fi\lthtmllogmath\lthtmlsetinline}
\newcommand\lthtmlinlinemathZ{\egroup\expandafter\ifdim\dp\sizebox>0pt %
  \expandafter\centerinlinemath\fi\lthtmllogmath\lthtmlsetmath}
\newcommand\lthtmlindisplaymathZ{\egroup %
  \centerinlinemath\lthtmllogmath\lthtmlsetmath}
\def\lthtmlsetinline{\hbox{\vrule width.1em \vtop{\vbox{%
  \kern.1em\copy\sizebox}\ifdim\dp\sizebox>0pt\kern.1em\else\kern.3pt\fi
  \ifdim\hsize>\wd\sizebox \hrule depth1pt\fi}}}
\def\lthtmlsetmath{\hbox{\vrule width.1em\kern-.05em\vtop{\vbox{%
  \kern.1em\kern0.8 pt\hbox{\hglue.17em\copy\sizebox\hglue0.8 pt}}\kern.3pt%
  \ifdim\dp\sizebox>0pt\kern.1em\fi \kern0.8 pt%
  \ifdim\hsize>\wd\sizebox \hrule depth1pt\fi}}}
\def\centerinlinemath{%
  \dimen1=\ifdim\ht\sizebox<\dp\sizebox \dp\sizebox\else\ht\sizebox\fi
  \advance\dimen1by.5pt \vrule width0pt height\dimen1 depth\dimen1 
 \dp\sizebox=\dimen1\ht\sizebox=\dimen1\relax}

\def\lthtmlcheckvsize{\ifdim\ht\sizebox<\vsize 
  \ifdim\wd\sizebox<\hsize\expandafter\hfill\fi \expandafter\vfill
  \else\expandafter\vss\fi}%
\providecommand{\selectlanguage}[1]{}%
\makeatletter \tracingstats = 1 


\begin{document}
\pagestyle{empty}\thispagestyle{empty}\lthtmltypeout{}%
\lthtmltypeout{latex2htmlLength hsize=\the\hsize}\lthtmltypeout{}%
\lthtmltypeout{latex2htmlLength vsize=\the\vsize}\lthtmltypeout{}%
\lthtmltypeout{latex2htmlLength hoffset=\the\hoffset}\lthtmltypeout{}%
\lthtmltypeout{latex2htmlLength voffset=\the\voffset}\lthtmltypeout{}%
\lthtmltypeout{latex2htmlLength topmargin=\the\topmargin}\lthtmltypeout{}%
\lthtmltypeout{latex2htmlLength topskip=\the\topskip}\lthtmltypeout{}%
\lthtmltypeout{latex2htmlLength headheight=\the\headheight}\lthtmltypeout{}%
\lthtmltypeout{latex2htmlLength headsep=\the\headsep}\lthtmltypeout{}%
\lthtmltypeout{latex2htmlLength parskip=\the\parskip}\lthtmltypeout{}%
\lthtmltypeout{latex2htmlLength oddsidemargin=\the\oddsidemargin}\lthtmltypeout{}%
\makeatletter
\if@twoside\lthtmltypeout{latex2htmlLength evensidemargin=\the\evensidemargin}%
\else\lthtmltypeout{latex2htmlLength evensidemargin=\the\oddsidemargin}\fi%
\lthtmltypeout{}%
\makeatother
\setcounter{page}{1}
\onecolumn

% !!! IMAGES START HERE !!!

{\newpage\clearpage
\lthtmlinlinemathA{tex2html_wrap_inline667}%
$y$%
\lthtmlinlinemathZ
\lthtmlcheckvsize\clearpage}

{\newpage\clearpage
\lthtmlinlinemathA{tex2html_wrap_inline669}%
$p$%
\lthtmlinlinemathZ
\lthtmlcheckvsize\clearpage}

{\newpage\clearpage
\lthtmlinlinemathA{tex2html_wrap_inline671}%
$x$%
\lthtmlinlinemathZ
\lthtmlcheckvsize\clearpage}

{\newpage\clearpage
\lthtmlinlinemathA{tex2html_wrap_inline673}%
$\beta_1'x,\ldots,\beta_d'x$%
\lthtmlinlinemathZ
\lthtmlcheckvsize\clearpage}

{\newpage\clearpage
\lthtmlinlinemathA{tex2html_wrap_inline675}%
$d$%
\lthtmlinlinemathZ
\lthtmlcheckvsize\clearpage}

{\newpage\clearpage
\lthtmlinlinemathA{tex2html_wrap_inline679}%
$d=1$%
\lthtmlinlinemathZ
\lthtmlcheckvsize\clearpage}

{\newpage\clearpage
\lthtmlinlinemathA{tex2html_wrap_inline683}%
$\beta_1'Xx$%
\lthtmlinlinemathZ
\lthtmlcheckvsize\clearpage}

{\newpage\clearpage
\lthtmlinlinemathA{tex2html_wrap_inline685}%
$d=2$%
\lthtmlinlinemathZ
\lthtmlcheckvsize\clearpage}

{\newpage\clearpage
\lthtmlinlinemathA{tex2html_wrap_inline689}%
$\beta_1\ldots,\beta_d$%
\lthtmlinlinemathZ
\lthtmlcheckvsize\clearpage}

{\newpage\clearpage
\lthtmlinlinemathA{tex2html_wrap_inline693}%
$\beta_1,\ldots,\beta_d$%
\lthtmlinlinemathZ
\lthtmlcheckvsize\clearpage}

\stepcounter{section}
{\newpage\clearpage
\lthtmlinlinemathA{tex2html_wrap_inline697}%
$k$%
\lthtmlinlinemathZ
\lthtmlcheckvsize\clearpage}

{\newpage\clearpage
\lthtmlinlinemathA{tex2html_wrap_inline699}%
$k=1$%
\lthtmlinlinemathZ
\lthtmlcheckvsize\clearpage}

{\newpage\clearpage
\lthtmlinlinemathA{tex2html_wrap_inline705}%
$F(y|x)$%
\lthtmlinlinemathZ
\lthtmlcheckvsize\clearpage}

{\newpage\clearpage
\lthtmlinlinemathA{tex2html_wrap_inline709}%
$F$%
\lthtmlinlinemathZ
\lthtmlcheckvsize\clearpage}

{\newpage\clearpage
\lthtmlinlinemathA{tex2html_wrap_inline717}%
$p\times d$%
\lthtmlinlinemathZ
\lthtmlcheckvsize\clearpage}

{\newpage\clearpage
\lthtmlinlinemathA{tex2html_wrap_inline719}%
$B$%
\lthtmlinlinemathZ
\lthtmlcheckvsize\clearpage}

{\newpage\clearpage
\lthtmldisplayA{displaymath655}%
\begin{displaymath}
F(y|x) = F(y|B'x)

\end{displaymath}%
\lthtmldisplayZ
\lthtmlcheckvsize\clearpage}

{\newpage\clearpage
\lthtmlinlinemathA{tex2html_wrap_inline721}%
$B=I$%
\lthtmlinlinemathZ
\lthtmlcheckvsize\clearpage}

{\newpage\clearpage
\lthtmlinlinemathA{tex2html_wrap_inline723}%
$p\times p$%
\lthtmlinlinemathZ
\lthtmlcheckvsize\clearpage}

{\newpage\clearpage
\lthtmlinlinemathA{tex2html_wrap_inline729}%
$B^*=BA$%
\lthtmlinlinemathZ
\lthtmlcheckvsize\clearpage}

{\newpage\clearpage
\lthtmlinlinemathA{tex2html_wrap_inline731}%
$A$%
\lthtmlinlinemathZ
\lthtmlcheckvsize\clearpage}

{\newpage\clearpage
\lthtmlinlinemathA{tex2html_wrap_inline735}%
${\mathcal S}(B)$%
\lthtmlinlinemathZ
\lthtmlcheckvsize\clearpage}

{\newpage\clearpage
\lthtmlinlinemathA{tex2html_wrap_inline743}%
$(x_i,y_i)$%
\lthtmlinlinemathZ
\lthtmlcheckvsize\clearpage}

{\newpage\clearpage
\lthtmlinlinemathA{tex2html_wrap_inline745}%
$i=1,\ldots,n$%
\lthtmlinlinemathZ
\lthtmlcheckvsize\clearpage}

{\newpage\clearpage
\lthtmlinlinemathA{tex2html_wrap_inline747}%
$X$%
\lthtmlinlinemathZ
\lthtmlcheckvsize\clearpage}

{\newpage\clearpage
\lthtmlinlinemathA{tex2html_wrap_inline749}%
$Y$%
\lthtmlinlinemathZ
\lthtmlcheckvsize\clearpage}

{\newpage\clearpage
\lthtmlinlinemathA{tex2html_wrap_inline755}%
$k>1$%
\lthtmlinlinemathZ
\lthtmlcheckvsize\clearpage}

{\newpage\clearpage
\lthtmlinlinemathA{tex2html_wrap_inline757}%
$w_1,\ldots,w_n$%
\lthtmlinlinemathZ
\lthtmlcheckvsize\clearpage}

{\newpage\clearpage
\lthtmlinlinemathA{tex2html_wrap_inline759}%
$n$%
\lthtmlinlinemathZ
\lthtmlcheckvsize\clearpage}

{\newpage\clearpage
\lthtmlinlinemathA{tex2html_wrap_inline761}%
$w_i=1$%
\lthtmlinlinemathZ
\lthtmlcheckvsize\clearpage}

{\newpage\clearpage
\lthtmldisplayA{displaymath656}%
\begin{displaymath}
    Z =
    W^{1/2}(X - 1\bar{x}') \hat{\Sigma}^{-1/2}
    \end{displaymath}%
\lthtmldisplayZ
\lthtmlcheckvsize\clearpage}

{\newpage\clearpage
\lthtmlinlinemathA{tex2html_wrap_inline769}%
$\bar{x} = \sum w_ix_i/\sum w_i$%
\lthtmlinlinemathZ
\lthtmlcheckvsize\clearpage}

{\newpage\clearpage
\lthtmlinlinemathA{tex2html_wrap_inline771}%
$W = \mathrm{diag}(w_i)$%
\lthtmlinlinemathZ
\lthtmlcheckvsize\clearpage}

{\newpage\clearpage
\lthtmldisplayA{displaymath657}%
\begin{displaymath}
    \hat{\Sigma} = \frac{1}{n -1} (X - 1\bar{x}')'W(X - 1\bar{x}')
    \end{displaymath}%
\lthtmldisplayZ
\lthtmlcheckvsize\clearpage}

{\newpage\clearpage
\lthtmlinlinemathA{tex2html_wrap_inline773}%
$\hat{\Sigma}^{-1/2}$%
\lthtmlinlinemathZ
\lthtmlcheckvsize\clearpage}

{\newpage\clearpage
\lthtmlinlinemathA{tex2html_wrap_inline777}%
$\bar{x}$%
\lthtmlinlinemathZ
\lthtmlcheckvsize\clearpage}

{\newpage\clearpage
\lthtmlinlinemathA{tex2html_wrap_inline779}%
$Z$%
\lthtmlinlinemathZ
\lthtmlcheckvsize\clearpage}

{\newpage\clearpage
\lthtmlinlinemathA{tex2html_wrap_inline785}%
$\hat{M}$%
\lthtmlinlinemathZ
\lthtmlcheckvsize\clearpage}

{\newpage\clearpage
\lthtmlinlinemathA{tex2html_wrap_inline787}%
$M$%
\lthtmlinlinemathZ
\lthtmlcheckvsize\clearpage}

{\newpage\clearpage
\lthtmlinlinemathA{tex2html_wrap_inline789}%
${\mathcal S}(M)
    \subseteq {\mathcal S}(B)$%
\lthtmlinlinemathZ
\lthtmlcheckvsize\clearpage}

{\newpage\clearpage
\lthtmlinlinemathA{tex2html_wrap_inline797}%
$|\hat{\lambda}_1|\geq \ldots \geq |\hat{\lambda}_p|$%
\lthtmlinlinemathZ
\lthtmlcheckvsize\clearpage}

{\newpage\clearpage
\lthtmlinlinemathA{tex2html_wrap_inline801}%
$\hat{u}_1,\ldots,\hat{u}_p$%
\lthtmlinlinemathZ
\lthtmlcheckvsize\clearpage}

{\newpage\clearpage
\lthtmlinlinemathA{tex2html_wrap_inline805}%
$d=d_0$%
\lthtmlinlinemathZ
\lthtmlcheckvsize\clearpage}

{\newpage\clearpage
\lthtmlinlinemathA{tex2html_wrap_inline807}%
$d >
    d_0$%
\lthtmlinlinemathZ
\lthtmlcheckvsize\clearpage}

{\newpage\clearpage
\lthtmldisplayA{displaymath658}%
\begin{displaymath}
    \Lambda_{d_0} = n \hat{c} \sum_{j=d_0+1}^p |\hat{\lambda}_j|^{\nu}
    \end{displaymath}%
\lthtmldisplayZ
\lthtmlcheckvsize\clearpage}

{\newpage\clearpage
\lthtmlinlinemathA{tex2html_wrap_inline809}%
$\hat{c}$%
\lthtmlinlinemathZ
\lthtmlcheckvsize\clearpage}

{\newpage\clearpage
\lthtmlinlinemathA{tex2html_wrap_inline811}%
$\nu$%
\lthtmlinlinemathZ
\lthtmlcheckvsize\clearpage}

{\newpage\clearpage
\lthtmlinlinemathA{tex2html_wrap_inline821}%
$\Re^n$%
\lthtmlinlinemathZ
\lthtmlcheckvsize\clearpage}

{\newpage\clearpage
\lthtmlinlinemathA{tex2html_wrap_inline823}%
$Z\hat{u}_1,\ldots,Z\hat{u}_d$%
\lthtmlinlinemathZ
\lthtmlcheckvsize\clearpage}

\stepcounter{section}
{\newpage\clearpage
\lthtmlinlinemathA{tex2html_wrap_inline843}%
$k=2$%
\lthtmlinlinemathZ
\lthtmlcheckvsize\clearpage}

{\newpage\clearpage
\lthtmlinlinemathA{tex2html_wrap_inline845}%
$\sqrt{8}
\approx 3$%
\lthtmlinlinemathZ
\lthtmlcheckvsize\clearpage}

{\newpage\clearpage
\lthtmlinlinemathA{tex2html_wrap_inline847}%
$3\times 3=9$%
\lthtmlinlinemathZ
\lthtmlcheckvsize\clearpage}

\stepcounter{section}
\stepcounter{subsection}
{\newpage\clearpage
\lthtmlinlinemathA{tex2html_wrap_inline855}%
$\mbox{E}(A'x|B'x)$%
\lthtmlinlinemathZ
\lthtmlcheckvsize\clearpage}

{\newpage\clearpage
\lthtmlinlinemathA{tex2html_wrap_inline857}%
$B'x$%
\lthtmlinlinemathZ
\lthtmlcheckvsize\clearpage}

{\newpage\clearpage
\lthtmlinlinemathA{tex2html_wrap_inline863}%
$F(y|x) = F(y|B'x)$%
\lthtmlinlinemathZ
\lthtmlcheckvsize\clearpage}

{\newpage\clearpage
\lthtmlinlinemathA{tex2html_wrap_inline867}%
$F(z|y) \subseteq
{\mathcal S}(B)$%
\lthtmlinlinemathZ
\lthtmlcheckvsize\clearpage}

{\newpage\clearpage
\lthtmlinlinemathA{tex2html_wrap_inline869}%
$F(z|y)$%
\lthtmlinlinemathZ
\lthtmlcheckvsize\clearpage}

{\newpage\clearpage
\lthtmlinlinemathA{tex2html_wrap_inline873}%
$h$%
\lthtmlinlinemathZ
\lthtmlcheckvsize\clearpage}

{\newpage\clearpage
\lthtmlinlinemathA{tex2html_wrap_inline879}%
$Y_1 \times Y_2 \ldots \times Y_k$%
\lthtmlinlinemathZ
\lthtmlcheckvsize\clearpage}

{\newpage\clearpage
\lthtmlinlinemathA{tex2html_wrap_inline883}%
$k = 3$%
\lthtmlinlinemathZ
\lthtmlcheckvsize\clearpage}

{\newpage\clearpage
\lthtmlinlinemathA{tex2html_wrap_inline885}%
$Y_1$%
\lthtmlinlinemathZ
\lthtmlcheckvsize\clearpage}

{\newpage\clearpage
\lthtmlinlinemathA{tex2html_wrap_inline887}%
$Y_2$%
\lthtmlinlinemathZ
\lthtmlcheckvsize\clearpage}

{\newpage\clearpage
\lthtmlinlinemathA{tex2html_wrap_inline889}%
$Y_3$%
\lthtmlinlinemathZ
\lthtmlcheckvsize\clearpage}

{\newpage\clearpage
\lthtmlinlinemathA{tex2html_wrap_inline891}%
$h=2\times3\times4=24$%
\lthtmlinlinemathZ
\lthtmlcheckvsize\clearpage}

{\newpage\clearpage
\lthtmlinlinemathA{tex2html_wrap_inline899}%
$h \times p$%
\lthtmlinlinemathZ
\lthtmlcheckvsize\clearpage}

{\newpage\clearpage
\lthtmlinlinemathA{tex2html_wrap_inline901}%
$i$%
\lthtmlinlinemathZ
\lthtmlcheckvsize\clearpage}

{\newpage\clearpage
\lthtmlinlinemathA{tex2html_wrap_inline909}%
$\mbox{E}(Z|Y)$%
\lthtmlinlinemathZ
\lthtmlcheckvsize\clearpage}

{\newpage\clearpage
\lthtmlinlinemathA{tex2html_wrap_inline923}%
$d=0$%
\lthtmlinlinemathZ
\lthtmlcheckvsize\clearpage}

{\newpage\clearpage
\lthtmlinlinemathA{tex2html_wrap_inline925}%
$d>0$%
\lthtmlinlinemathZ
\lthtmlcheckvsize\clearpage}

{\newpage\clearpage
\lthtmlinlinemathA{tex2html_wrap_inline933}%
$d>1$%
\lthtmlinlinemathZ
\lthtmlcheckvsize\clearpage}

{\newpage\clearpage
\lthtmlinlinemathA{tex2html_wrap_inline937}%
$0.001$%
\lthtmlinlinemathZ
\lthtmlcheckvsize\clearpage}

{\newpage\clearpage
\lthtmlinlinemathA{tex2html_wrap_inline943}%
$d>2$%
\lthtmlinlinemathZ
\lthtmlcheckvsize\clearpage}

{\newpage\clearpage
\lthtmlinlinemathA{tex2html_wrap_inline947}%
$0.31$%
\lthtmlinlinemathZ
\lthtmlcheckvsize\clearpage}

\stepcounter{subsection}
{\newpage\clearpage
\lthtmlinlinemathA{tex2html_wrap_inline955}%
$C_i$%
\lthtmlinlinemathZ
\lthtmlcheckvsize\clearpage}

{\newpage\clearpage
\lthtmldisplayA{displaymath659}%
\begin{displaymath}
\hat{M} = \frac{1}{n}\sum g_i(I-C_i)^2
\end{displaymath}%
\lthtmldisplayZ
\lthtmlcheckvsize\clearpage}

{\newpage\clearpage
\lthtmlinlinemathA{tex2html_wrap_inline961}%
$g_i$%
\lthtmlinlinemathZ
\lthtmlcheckvsize\clearpage}

\stepcounter{subsection}
{\newpage\clearpage
\lthtmldisplayA{displaymath660}%
\begin{displaymath}
\hat{M} = \frac{1}{n}\sum_{i=1}^n w_i f_i z_iz_i'
\end{displaymath}%
\lthtmldisplayZ
\lthtmlcheckvsize\clearpage}

{\newpage\clearpage
\lthtmlinlinemathA{tex2html_wrap_inline967}%
$f_i$%
\lthtmlinlinemathZ
\lthtmlcheckvsize\clearpage}

{\newpage\clearpage
\lthtmlinlinemathA{tex2html_wrap_inline969}%
$y_i$%
\lthtmlinlinemathZ
\lthtmlcheckvsize\clearpage}

{\newpage\clearpage
\lthtmlinlinemathA{tex2html_wrap_inline973}%
$w_i$%
\lthtmlinlinemathZ
\lthtmlcheckvsize\clearpage}

{\newpage\clearpage
\lthtmlinlinemathA{tex2html_wrap_inline977}%
$\sum w_i = n$%
\lthtmlinlinemathZ
\lthtmlcheckvsize\clearpage}

\stepcounter{subsection}
{\newpage\clearpage
\lthtmlinlinemathA{tex2html_wrap_inline997}%
$(i,j)$%
\lthtmlinlinemathZ
\lthtmlcheckvsize\clearpage}

{\newpage\clearpage
\lthtmlinlinemathA{tex2html_wrap_inline999}%
$x_ix_j$%
\lthtmlinlinemathZ
\lthtmlcheckvsize\clearpage}

\stepcounter{subsection}
\stepcounter{subsection}
{\newpage\clearpage
\lthtmlfigureA{table185}%
\begin{table}
\hrule 

\small\begin{verbatim}

#####################################################################
#     Sliced Inverse Regression
#####################################################################

dr.fit.M.sir <-function(object,z,y,w=NULL,nslices=NULL,
                        slice.info=NULL,...) {
# get slice information
    h <- if (!is.null(nslices)) nslices else max(8, NCOL(z)+3)
    slices<- if(is.null(slice.info)) dr.slices(y,h) else slice.info
# initialize slice means matrix
    zmeans <- matrix(0,slices$nslices,NCOL(z))
    slice.weight <- slices$nslices
# make sure weights add to n
    wts <- if(is.null(w)) rep(1,NROW(z)) else NROW(z) * w /sum(w)
# compute weighted means within slice (weights always add to n)
    wmean <- function (x, wts) { sum(x * wts) / sum (wts) }
    for (j in 1:slices$nslices){
      sel <- slices$slice.indicator==j
      zmeans[j,]<- apply(z[sel,],2,wmean,wts[sel])
      slice.weight[j]<-sum(wts[sel])}
# get M matrix for sir
    M <- t(zmeans) %*% apply(zmeans,2,"*",slice.weight)/ sum(slice.weight)
    return (list (M=M,slice.info=slices))
}\end{verbatim}

\hrule 
\normalsize 
\end{table}%
\lthtmlfigureZ
\lthtmlcheckvsize\clearpage}

{\newpage\clearpage
\lthtmlinlinemathA{tex2html_wrap_inline1007}%
$n \times p$%
\lthtmlinlinemathZ
\lthtmlcheckvsize\clearpage}

{\newpage\clearpage
\lthtmlfigureA{table208}%
\begin{table}
\hrule 

\small\begin{verbatim}

dr.test.sir<-function(object,nd) {
#compute the sir test statistic for the first nd directions
    e<-sort(object$evalues)
    p<-length(object$evalues)
    n<-object$cases
    st<-df<-pv<-0
    nt <- min(p,nd)
    for (i in 0:nt-1)
      {st[i+1]<-n*(p-i)*mean(e[seq(1,p-i)])
       df[i+1]<-(p-i)*(object$slice.info$nslices-i-1)
       pv[i+1]<-1-pchisq(st[i+1],df[i+1])
      }
    z<-data.frame(cbind(st,df,pv))
    rr<-paste(0:(nt-1),"D vs >= ",1:nt,"D",sep="")
    dimnames(z)<-list(rr,c("Stat","df","p-value"))
    z
}\end{verbatim}

\normalsize\hrule 
\end{table}%
\lthtmlfigureZ
\lthtmlcheckvsize\clearpage}

\stepcounter{section}
\stepcounter{section}
\stepcounter{section}
{\newpage\clearpage
\lthtmlinlinemathA{tex2html_wrap_inline1019}%
$m$%
\lthtmlinlinemathZ
\lthtmlcheckvsize\clearpage}

{\newpage\clearpage
\lthtmlinlinemathA{tex2html_wrap_inline1021}%
$S$%
\lthtmlinlinemathZ
\lthtmlcheckvsize\clearpage}

{\newpage\clearpage
\lthtmlinlinemathA{tex2html_wrap_inline1035}%
$Z = (X - 1m')S^{-1/2}$%
\lthtmlinlinemathZ
\lthtmlcheckvsize\clearpage}

{\newpage\clearpage
\lthtmlinlinemathA{tex2html_wrap_inline1037}%
$\mbox{N}(m,S)$%
\lthtmlinlinemathZ
\lthtmlcheckvsize\clearpage}

{\newpage\clearpage
\lthtmlinlinemathA{tex2html_wrap_inline1041}%
$\mbox{N}(0,I)$%
\lthtmlinlinemathZ
\lthtmlcheckvsize\clearpage}

{\newpage\clearpage
\lthtmlinlinemathA{tex2html_wrap_inline1043}%
$b$%
\lthtmlinlinemathZ
\lthtmlcheckvsize\clearpage}

{\newpage\clearpage
\lthtmlinlinemathA{tex2html_wrap_inline1045}%
$\mbox{N}(0,\sigma^2I)$%
\lthtmlinlinemathZ
\lthtmlcheckvsize\clearpage}

\stepcounter{section}
{\newpage\clearpage
\lthtmlinlinemathA{tex2html_wrap_inline1059}%
$XU$%
\lthtmlinlinemathZ
\lthtmlcheckvsize\clearpage}

{\newpage\clearpage
\lthtmlinlinemathA{tex2html_wrap_inline1061}%
$U$%
\lthtmlinlinemathZ
\lthtmlcheckvsize\clearpage}

{\newpage\clearpage
\lthtmlinlinemathA{tex2html_wrap_inline1073}%
$^{1/p}$%
\lthtmlinlinemathZ
\lthtmlcheckvsize\clearpage}

{\newpage\clearpage
\lthtmlinlinemathA{tex2html_wrap_inline1075}%
$p=2$%
\lthtmlinlinemathZ
\lthtmlcheckvsize\clearpage}

{\newpage\clearpage
\lthtmlinlinemathA{tex2html_wrap_inline1077}%
$=8$%
\lthtmlinlinemathZ
\lthtmlcheckvsize\clearpage}

\stepcounter{section}
\stepcounter{section}
\stepcounter{section}
\stepcounter{section}
\stepcounter{section}
{\newpage\clearpage
\lthtmlfigureA{Description1781}%
\begin{Description}\relax
The function dr implements dimension reduction methods, including SIR, SAVE and pHd.\end{Description}%
\lthtmlfigureZ
\lthtmlcheckvsize\clearpage}

{\newpage\clearpage
\lthtmlfigureA{Usage1783}%
\begin{Usage}
\begin{verbatim}

dr(formula, data=list(), subset, weights, na.action=na.omit, method="sir", 
     contrasts=NULL,numdir=4, ...)\end{verbatim}
\end{Usage}%
\lthtmlfigureZ
\lthtmlcheckvsize\clearpage}

{\newpage\clearpage
\lthtmlfigureA{Arguments1787}%
\begin{Arguments}
\begin{ldescription}
\item[\code{formula}] a symbolic description of the model to be fit. The details of
the model are the same as for lm. 
\item[\code{data}] an optional data frame containing the variables in the model.
By default the variables are taken from the environment from
which `dr' is called.
\item[\code{subset}] an optional vector specifying a subset of observations to be
used in the fitting process.
\item[\code{weights}] an optional vector of weights to be used where appropriate.
\item[\code{na.action}] a function which indicates what should happen when the data
contain `NA's.  The default is `na.omit,' which will force
calculations on a complete subset of cases.
\item[\code{method}] This character string specifies the method of fitting.  ``sir"
specifies sliced inverse regression and ``save" specifies sliced
average variance estimation.  ``phdy" uses principal hessian
directions using the response as suggested by Li, and ``phdres" 
uses the LS residuals as suggested by Cook. Other methods may be
added
\item[\code{contrasts}] an optional list. See the `contrasts.arg' of
`model.matrix.default'.
\item[\code{numdir}] Maximum number of directions to consider
\item[\code{...}] additional items that may be required or permitted by some 
methods.
nslices is the number of slices used by sir and save.
\item[\code{object}] 
\end{ldescription}
 A dr object created by a call to dr.\end{Arguments}%
\lthtmlfigureZ
\lthtmlcheckvsize\clearpage}

{\newpage\clearpage
\lthtmlfigureA{Details1801}%
\begin{Details}\relax
The general regression problem studies \eqn{F(y|x)}{}, the conditional
distribution of a response \eqn{y}{} given a set of predictors \eqn{x}{}.  
This function provides methods for estimating the dimension and central
subspace of a general regression problem.  That is, we want to find a 
\eqn{p \times d}{p by d} matrix \eqn{B}{} such that 
\deqn{F(y|x)=F(y|B'x)}{}  
Both the dimension \eqn{d}{} and the subspace
\eqn{R(B)}{} are unknown.  These methods make few assumptions.  All the methods
available in this function estimate the unknowns by study of the inverse
problem, \eqn{F(x|y)}{}.  In each, a kernel matrix \eqn{M}{} is estimated such
that the column space of \eqn{M}{} should be close to the central subspace.
Eigenanalysis of \eqn{M}{} is then used to estimate the central subspace.
Objects created using this function have appropriate print, summary and plot
methods.
\par Weights can be used, essentially to specify the relative 
frequency of each case in the data.  Empirical weights that make 
the contours of the weighted sample closer to elliptical can be 
computed.  This will usually result in zero weight for some 
cases.  The function will set zero estimated weights to missing.
\par Several functions are provided that require a dr object as input.  
dr.permutation.tests uses a permutation test to obtain significance levels
for tests of dimension.  dr.coplot allows visualizing the results using a
coplot of either two selected directions conditioning on a third and using
color to mark the response, or the resonse versus one direction,
conditioning on a second direction.  plot.dr provides the default plot
method for dr objects, based on a scatterplot matrix.\end{Details}%
\lthtmlfigureZ
\lthtmlcheckvsize\clearpage}

{\newpage\clearpage
\lthtmlfigureA{Value1827}%
\begin{Value}
Unless \code{estimate.weights=T}, returns a list of items, including
\begin{ldescription}
\item[\code{M}] A matrix that depends on the method of computing.  The column space
of M should be close to the central subspace.
\item[\code{evalues}] The eigenvalues of M (or squared singular values if M is not
symmetric).
\item[\code{evectors}] The eigenvectors of M (or of M'M if M is not square and
symmetric) ordered according to the eigenvalues.
\item[\code{numdir}] The maximum number of directions to be found.  The output
value of numdir may be smaller than the input value.
\item[\code{ols.coef}] Estimated regression coefficients, excluding the intercept,
for the (weighted) LS fit.
\item[\code{ols.fit}] LS fitted values.
\item[\code{slice.info}] output from sir.slice, used by sir and save.
\item[\code{method}] the dimension reduction method used.
\end{ldescription}
\par Other returned values repeat quantities from input.\end{Value}%
\lthtmlfigureZ
\lthtmlcheckvsize\clearpage}

{\newpage\clearpage
\lthtmlfigureA{Author1840}%
\begin{Author}\relax
Sanford Weisberg, 
sandy@stat.umn.edu\end{Author}%
\lthtmlfigureZ
\lthtmlcheckvsize\clearpage}

{\newpage\clearpage
\lthtmlfigureA{References1842}%
\begin{References}\relax
The details of these methods are given by R. D. Cook 
(1998).  Regression Graphics.  New York:  Wiley.  Equivalent 
methods are also available in Arc, R. D. Cook and S. Weisberg 
(1999).  Applied Regression Including Computing and Graphics, New 
York:  Wiley, www.stat.umn.edu/arc.\end{References}%
\lthtmlfigureZ
\lthtmlcheckvsize\clearpage}

{\newpage\clearpage
\lthtmlfigureA{SeeAlso1844}%
\begin{SeeAlso}\relax
dr.permutation.test,dr.x,dr.y,dr.direction,dr.coplot,plot.dr\end{SeeAlso}%
\lthtmlfigureZ
\lthtmlcheckvsize\clearpage}

{\newpage\clearpage
\lthtmlfigureA{Examples1846}%
\begin{Examples}
\begin{ExampleCode}
library(dr)
data(ais)
attach(ais)  # the Australian athletes data
#fit dimension reduction using sir
m1 <- dr(LBM~Wt+Ht+RCC+WCC, method="sir", nslices = 8)
summary(m1)
\par # repeat, using save:
\par m2 <- update(m1,method="save")
summary(m2)
\par # repeat, using phd:
\par m3 <- update(m2, method="phdres")
summary(m3)
\end{ExampleCode}
\end{Examples}%
\lthtmlfigureZ
\lthtmlcheckvsize\clearpage}

{\newpage\clearpage
\lthtmlfigureA{Description1868}%
\begin{Description}\relax
This function computes a permutation test for dimension for any inverse 
regression fitting method.\end{Description}%
\lthtmlfigureZ
\lthtmlcheckvsize\clearpage}

{\newpage\clearpage
\lthtmlfigureA{Usage1870}%
\begin{Usage}
\begin{verbatim}

dr.permutation.test(object, npermute=50, numdir=object$numdir)\end{verbatim}
\end{Usage}%
\lthtmlfigureZ
\lthtmlcheckvsize\clearpage}

{\newpage\clearpage
\lthtmlfigureA{Arguments1874}%
\begin{Arguments}
\begin{ldescription}
\item[\code{object}] an inverse regression object created by dr
\item[\code{npermute}] number of permutations to compute, default is 50
\item[\code{numdir}] maximum permitted value of the dimension, with the default from
the object
\end{ldescription}
\end{Arguments}%
\lthtmlfigureZ
\lthtmlcheckvsize\clearpage}

{\newpage\clearpage
\lthtmlfigureA{Value1881}%
\begin{Value}
Returns an object of type 'dr.permutation.test' that can be printed or
summarized to give the summary of the test.\end{Value}%
\lthtmlfigureZ
\lthtmlcheckvsize\clearpage}

{\newpage\clearpage
\lthtmlfigureA{References1885}%
\begin{References}\relax
See www.stat.umn.edu/arc/addons.html, and then select the article
on inverse regression.\end{References}%
\lthtmlfigureZ
\lthtmlcheckvsize\clearpage}

{\newpage\clearpage
\lthtmlfigureA{SeeAlso1887}%
\begin{SeeAlso}\relax
\code{\Link{dr}}\end{SeeAlso}%
\lthtmlfigureZ
\lthtmlcheckvsize\clearpage}

{\newpage\clearpage
\lthtmlfigureA{Examples1890}%
\begin{Examples}
\begin{ExampleCode}
data(ais)
attach(ais)  # the Australian athletes data
#fit inverse regression using sir
m1 <- dr(LBM~Wt+Ht+RCC+WCC, method="sir", nslices = 8)
summary(m1)
dr.permutation.test(m1,npermute=100)
\end{ExampleCode}
\end{Examples}%
\lthtmlfigureZ
\lthtmlcheckvsize\clearpage}

{\newpage\clearpage
\lthtmlfigureA{Description1901}%
\begin{Description}\relax
These routines provide default plotting methods for dimension reduction regression.\end{Description}%
\lthtmlfigureZ
\lthtmlcheckvsize\clearpage}

{\newpage\clearpage
\lthtmlfigureA{Usage1903}%
\begin{Usage}
\begin{verbatim}

plot.dr(object, which=1:object$numdir, mark.by.y=F,plot.method=pairs, ...)
dr.coplot(object, which=1:object$numdir, mark.by.y=F, ...)\end{verbatim}
\end{Usage}%
\lthtmlfigureZ
\lthtmlcheckvsize\clearpage}

{\newpage\clearpage
\lthtmlfigureA{Arguments1907}%
\begin{Arguments}
\begin{ldescription}
\item[\code{object}] Any dimension reduction regression object
\item[\code{which}] Which directions to plot
\item[\code{mark.by.y}] if TRUE, use the response as a marking variable to color
points; if FALSE, use response in the plot
\item[\code{plot.method}] The default is to use the pairs or coplot method to
draw the plots.  If John Fox's car library is available, you can
substitute scatterplot.matrix for pairs.
\item[\code{...}] arguments passed to plot or coplot.  In particular, if the
car library is available, the argument panel=panel.car will add smoothers
to a coplot.
\end{ldescription}
\end{Arguments}%
\lthtmlfigureZ
\lthtmlcheckvsize\clearpage}

{\newpage\clearpage
\lthtmlfigureA{Value1916}%
\begin{Value}
Produces a scatterplot matrix (plot) or coplot (dr.coplot) of the specified
directions in an dimension reduction regression\end{Value}%
\lthtmlfigureZ
\lthtmlcheckvsize\clearpage}

{\newpage\clearpage
\lthtmlfigureA{References1920}%
\begin{References}\relax
Cook, R. D. and Weisberg, S. (1999).  Applied Regression
Including Computing and Graphics.  New York:  Wiley.\end{References}%
\lthtmlfigureZ
\lthtmlcheckvsize\clearpage}

{\newpage\clearpage
\lthtmlfigureA{Examples1922}%
\begin{Examples}
\begin{ExampleCode}
##---- Should be DIRECTLY executable !! ----
##-- ==>  Define data, use random,
##--         or do  help(data=index)  for the standard data sets.
data(ais)
attach(ais)
i1<-dr(LBM~Ht+Wt+RCC+WCC)
plot(i1)
dr.coplot(i1,mark.by.y=TRUE)
\end{ExampleCode}
\end{Examples}%
\lthtmlfigureZ
\lthtmlcheckvsize\clearpage}

{\newpage\clearpage
\lthtmlfigureA{Description1932}%
\begin{Description}\relax
This function draws several 2D views of a 3D plot, sort of like a spinning
plot.\end{Description}%
\lthtmlfigureZ
\lthtmlcheckvsize\clearpage}

{\newpage\clearpage
\lthtmlfigureA{Usage1934}%
\begin{Usage}
\begin{verbatim}

rotplot(x, y, theta=seq(0, pi/2, length = 9), ...)\end{verbatim}
\end{Usage}%
\lthtmlfigureZ
\lthtmlcheckvsize\clearpage}

{\newpage\clearpage
\lthtmlfigureA{Arguments1938}%
\begin{Arguments}
\begin{ldescription}
\item[\code{x}] a matrix with 2 columns giving the horizontal axes of the full 3D
plot.
\item[\code{y}] the vertical axis of the 3D plot.
\item[\code{theta}] a list of rotation angles
\item[\code{...}] additional arguments passed to coplot
\end{ldescription}
\end{Arguments}%
\lthtmlfigureZ
\lthtmlcheckvsize\clearpage}

{\newpage\clearpage
\lthtmlfigureA{Details1946}%
\begin{Details}\relax
For each value of theta, draw the plot of cos(theta)*x[,1]+sin(theta)*x[,2]
versus y.\end{Details}%
\lthtmlfigureZ
\lthtmlcheckvsize\clearpage}

{\newpage\clearpage
\lthtmlfigureA{Value1948}%
\begin{Value}
returns a graph object.\end{Value}%
\lthtmlfigureZ
\lthtmlcheckvsize\clearpage}

{\newpage\clearpage
\lthtmlfigureA{Examples1952}%
\begin{Examples}
\begin{ExampleCode}
 data(ais)
 attach(ais)
 m1 <- dr(LBM ~ Ht + Wt + WCC)  
 rotplot(dr.direction(m1,which=1:2),dr.y(m1),col=markby(Sex))
  \end{ExampleCode}
\end{Examples}%
\lthtmlfigureZ
\lthtmlcheckvsize\clearpage}

{\newpage\clearpage
\lthtmlfigureA{Description1964}%
\begin{Description}\relax
dr.x returns the matrix of data constructed from the formula for a
dimension reduction regression.  dr.y returns the response.\end{Description}%
\lthtmlfigureZ
\lthtmlcheckvsize\clearpage}

{\newpage\clearpage
\lthtmlfigureA{Usage1966}%
\begin{Usage}
\begin{verbatim}

dr.x(object)
dr.y(object)
dr.z(x,weights=NULL,center=TRUE,rotate=TRUE,decomp="svd")\end{verbatim}
\end{Usage}%
\lthtmlfigureZ
\lthtmlcheckvsize\clearpage}

{\newpage\clearpage
\lthtmlfigureA{Arguments1970}%
\begin{Arguments}
\begin{ldescription}
\item[\code{object}] An dimension reduction regression object
\item[\code{x}] 
\item[\code{weights}] 
\item[\code{center}] 
\item[\code{rotate}] 
\item[\code{decomp}] 
\end{ldescription}
 Decomposition to be used in computing the rotation; the
default is "svd".\end{Arguments}%
\lthtmlfigureZ
\lthtmlcheckvsize\clearpage}

{\newpage\clearpage
\lthtmlfigureA{Value1980}%
\begin{Value}
dr.x returns an \eqn{n \times p}{n by p} matrix of terms, excluding the 
intercept, constructed from the dimension reduction regression object.  dr.y returns the
response. dr.z returns a possibly centered and scaled version of x.\end{Value}%
\lthtmlfigureZ
\lthtmlcheckvsize\clearpage}

{\newpage\clearpage
\lthtmlfigureA{Author1984}%
\begin{Author}\relax
Sanford Weisberg, <sandy@stat.umn.edu>\end{Author}%
\lthtmlfigureZ
\lthtmlcheckvsize\clearpage}

{\newpage\clearpage
\lthtmlfigureA{SeeAlso1986}%
\begin{SeeAlso}\relax
dr\end{SeeAlso}%
\lthtmlfigureZ
\lthtmlcheckvsize\clearpage}

{\newpage\clearpage
\lthtmlfigureA{Examples1988}%
\begin{Examples}
\begin{ExampleCode}
data(ais)
attach(ais)
m1 <- dr(LBM~Ht+Wt+RCC+WCC)
dr.x(m1)
dr.y(m1)
\end{ExampleCode}
\end{Examples}%
\lthtmlfigureZ
\lthtmlcheckvsize\clearpage}

{\newpage\clearpage
\lthtmlfigureA{Description1998}%
\begin{Description}\relax
These functions estimate weights to apply to the rows of a data matrix to
make the resulting weighted matrix as close to multivariate normality as
possible.\end{Description}%
\lthtmlfigureZ
\lthtmlcheckvsize\clearpage}

{\newpage\clearpage
\lthtmlfigureA{Usage2000}%
\begin{Usage}
\begin{verbatim}

dr.weights(formula,...)
dr.estimate.weights(object, sigma=1, covmethod="mve", nsamples=NULL, ...)
robust.center.scale(x, ... )\end{verbatim}
\end{Usage}%
\lthtmlfigureZ
\lthtmlcheckvsize\clearpage}

{\newpage\clearpage
\lthtmlfigureA{Arguments2004}%
\begin{Arguments}
\begin{ldescription}
\item[\code{object}] a dimension reduction regression object name, or an n by p matrix
\item[\code{sigma}] A tuning parameter, with default 1, usually in the range .2
to 1.0
\item[\code{covmethod}] covmethod is passed as the argument \code{method} 
to the function \code{cov.rob} in the package
\code{lqs} which is required.  The choices are \code{"classical"},
\code{"mve"} and \code{"mcd"}.  This probably will not work with Splus.
If classical is selected, the usual estimate of the covariance matrix is
used, but the center is the medians, not the means.
\item[\code{nsamples}] The weights are determined by random sampling from a
data-determined normal distribution.  This controls the number of samples
\item[\code{x}] An \eqn{n \times p}{n by p} data matrix with no missing values
\item[\code{...}] Additional args are passed to \code{cov.rob}
\end{ldescription}
\end{Arguments}%
\lthtmlfigureZ
\lthtmlcheckvsize\clearpage}

{\newpage\clearpage
\lthtmlfigureA{Details2023}%
\begin{Details}\relax
The basic outline is:  (1) Estimate a mean m and covariance matrix S using a
possibly robust method; (2) For each iteration, obtain a random vector
from N(m,sigma*S).  Add 1 to a counter for observation i if the i-th row
of the data matrix is closest to the random vector; (3) return as weights
the sample faction allocated to each observation.  If you set the keyword
\code{weights.only} to \code{T} on the call to \code{dr}, then only the
list of weights will be returned.\end{Details}%
\lthtmlfigureZ
\lthtmlcheckvsize\clearpage}

{\newpage\clearpage
\lthtmlfigureA{Value2028}%
\begin{Value}
Returns a list of n weights, some of which may be zero.\end{Value}%
\lthtmlfigureZ
\lthtmlcheckvsize\clearpage}

{\newpage\clearpage
\lthtmlfigureA{References2032}%
\begin{References}\relax
R. D. Cook and C. Nachtsheim (1994), Reweighting to achieve
elliptically contoured predictors in regression.  Journal of the American
Statistical Association, 89, 592--599.\end{References}%
\lthtmlfigureZ
\lthtmlcheckvsize\clearpage}

{\newpage\clearpage
\lthtmlfigureA{SeeAlso2034}%
\begin{SeeAlso}\relax
SEE ALSO \code{\Link{lqs}},\code{\Link{rob.cov}}\end{SeeAlso}%
\lthtmlfigureZ
\lthtmlcheckvsize\clearpage}

{\newpage\clearpage
\lthtmlfigureA{Examples2038}%
\begin{Examples}
\begin{ExampleCode}
\end{ExampleCode}
\end{Examples}%
\lthtmlfigureZ
\lthtmlcheckvsize\clearpage}

{\newpage\clearpage
\lthtmlfigureA{Description2057}%
\begin{Description}\relax
Compute a multivariate location and scale estimate with a high
breakdown point -- this can be thought of as estimating the mean and
covariance of the \code{good} part of the data. \code{cov.mve} and
\code{cov.mcd} are compatibility wrappers.\end{Description}%
\lthtmlfigureZ
\lthtmlcheckvsize\clearpage}

{\newpage\clearpage
\lthtmlfigureA{Usage2062}%
\begin{Usage}
\begin{verbatim}

cov.rob(x, cor = FALSE, quantile.used = floor((n + p + 1)/2),
        method = c("mve", "mcd", "classical"), nsamp = "best", seed)
cov.mve(x, cor = FALSE, quantile.used = floor((n + p + 1)/2),
        nsamp = "best", seed)
cov.mcd(x, cor = FALSE, quantile.used = floor((n + p + 1)/2),
        nsamp = "best", seed)\end{verbatim}
\end{Usage}%
\lthtmlfigureZ
\lthtmlcheckvsize\clearpage}

{\newpage\clearpage
\lthtmlfigureA{Arguments2066}%
\begin{Arguments}
\begin{ldescription}
\item[\code{x}] a matrix or data frame.
\par\item[\code{cor}] should the returned result include a correlation matrix?
\par\item[\code{quantile.used}] the minimum number of the data points regarded as \code{good} points.
\par\item[\code{method}] the method to be used -- minimum volume ellipsoid, minimum
covariance determinant or classical product-moment. Using
\code{cov.mve} or \code{cov.mcd} forces \code{mve} or \code{mcd}
respectively.
\par\item[\code{nsamp}] the number of samples or \code{"best"} or \code{"exact"} or
\code{"sample"}.
If \code{"sample"} the number chosen is \code{min(5*p, 3000)}, taken
from Rousseeuw and Hubert (1997). If \code{"best"} exhaustive
enumeration is done up to 5000 samples: if \code{"exact"}
exhaustive enumeration will be attempted however many samples are needed.
\par\item[\code{seed}] the seed to be used for random sampling: see \code{\Link{RNGkind}}. The
current value of \code{.Random.seed} will be preserved if it is set.
\par\end{ldescription}
\end{Arguments}%
\lthtmlfigureZ
\lthtmlcheckvsize\clearpage}

{\newpage\clearpage
\lthtmlfigureA{Details2090}%
\begin{Details}\relax
For method \code{"mve"}, an approximate search is made of a subset of
size \code{quantile.used} with an enclosing ellipsoid of smallest volume; in
method \code{"mcd"} it is the volume of the Gaussian confidence
ellipsoid, equivalently the determinant of the classical covariance
matrix, that is minimized. The mean of the subset provides a first
estimate of the location, and the rescaled covariance matrix a first
estimate of scatter. The Mahalanobis distances of all the points from
the location estimate for this covariance matrix are calculated, and
those points within the 97.5\% point under Gaussian assumptions are
declared to be \code{good}. The final estimates are the mean and rescaled
covariance of the \code{good} points.
\par The rescaling is by the appropriate percentile under Gaussian data; in
addition the first covariance matrix has an \emph{ad hoc} finite-sample
correction given by Marazzi.
\par For method \code{"mve"} the search is made over ellipsoids determined
by the covariance matrix of \code{p} of the data points. For method
\code{"mcd"} an additional improvement step suggested by Rousseeuw and
van Driessen (1997) is used, in which once a subset of size
\code{quantile.used} is selected, an ellipsoid based on its covariance
is tested (as this will have no larger a determinant, and may be smaller).\end{Details}%
\lthtmlfigureZ
\lthtmlcheckvsize\clearpage}

{\newpage\clearpage
\lthtmlfigureA{Value2102}%
\begin{Value}
A list with components
\par\begin{ldescription}
\item[\code{center}] the final estimate of location.
\par\item[\code{cov}] the final estimate of scatter.
\par\item[\code{cor}] (only is \code{cor = TRUE}) the estimate of the correlation
matrix.
\par\item[\code{sing}] message giving number of singular samples out of total
\par\item[\code{crit}] the value of the criterion on log scale. For MCD this is
the determinant, and for MVE it is proportional to the volume.
\par\item[\code{best}] the subset used. For MVE the best sample, for MCD the best
set of size \code{quantile.used}.
\par\item[\code{n.obs}] total number of observations.
\par\end{ldescription}
\end{Value}%
\lthtmlfigureZ
\lthtmlcheckvsize\clearpage}

{\newpage\clearpage
\lthtmlfigureA{Author2115}%
\begin{Author}\relax
B.D. Ripley\end{Author}%
\lthtmlfigureZ
\lthtmlcheckvsize\clearpage}

{\newpage\clearpage
\lthtmlfigureA{References2117}%
\begin{References}\relax
P. J. Rousseeuw and A. M. Leroy (1987) 
\emph{Robust Regression and Outlier Detection.}
Wiley.
\par A. Marazzi (1993) 
\emph{Algorithms, Routines and S Functions for Robust Statistics.}
Wadsworth and Brooks/Cole. 
\par P. J. Rousseeuw and B. C. van Zomeren (1990) Unmasking
multivariate outliers and leverage points, 
\emph{Journal of the American Statistical Association}, \bold{85}, 633--639.
\par P. J. Rousseeuw and K. van Driessen (1999) A fast algorithm for the
minimum covariance determinant estimator. \emph{Technometrics}
\bold{41}, 212--223.
\par P. Rousseeuw and M. Hubert (1997) Recent developments in PROGRESS. In
\emph{L1-Statistical Procedures and Related Topics }
ed Y. Dodge, IMS Lecture Notes volume \bold{31}, pp. 201--214.\end{References}%
\lthtmlfigureZ
\lthtmlcheckvsize\clearpage}

{\newpage\clearpage
\lthtmlfigureA{SeeAlso2127}%
\begin{SeeAlso}\relax
\code{\Link{lqs}}\end{SeeAlso}%
\lthtmlfigureZ
\lthtmlcheckvsize\clearpage}

{\newpage\clearpage
\lthtmlfigureA{Examples2130}%
\begin{Examples}
\begin{ExampleCode}
data(stackloss)
set.seed(123)
cov.rob(stackloss)
cov.rob(stack.x, method = "mcd", nsamp = "exact")
\end{ExampleCode}
\end{Examples}%
\lthtmlfigureZ
\lthtmlcheckvsize\clearpage}


\end{document}
