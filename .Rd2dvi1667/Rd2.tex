\documentclass[letterpaper]{book}
\usepackage[ae]{Rd}
\usepackage{makeidx}
\makeindex
\begin{document}
\chapter*{}
\begin{center}
{\textbf{\huge \R{} documentation}} \par\bigskip{{\Large of all in \file{./man}}}
\par\medskip{\large \today}
\end{center}
\Rdcontents{\R{} topics documented:}
\Header{ais}{}
\keyword{datasets}{ais}
\begin{Description}\relax
Data on 102 male and 100 female athletes collected at the Australian
Institute of Sport, courtesy of Richard Telford and Ross Cunningham.\end{Description}
\begin{Format}\relax
This data frame contains the following columns:
\describe{
\item[Sex] (0 = male or 1 = female)

\item[Ht] height (cm)

\item[Wt] weight (kg)

\item[LBM] lean body mass

\item[RCC] red cell count

\item[WCC] white cell count

\item[Hc] Hematocrit

\item[Hg] Hemoglobin

}\end{Format}
\begin{Details}\relax
\end{Details}
\begin{Source}\relax
R. D. Cook and S. Weisberg (1999). \emph{Applied
Statistics Including Computing and Graphics}.  New York:  Wiley.\end{Source}
\begin{Examples}
\begin{ExampleCode}
data(ais)
\end{ExampleCode}
\end{Examples}

\Header{givens.rotation}{Create givens rotation matrix}
\keyword{~keyword}{givens.rotation}
\begin{Description}\relax
For a given angle theta, returns a p by p Givens rotation matrix.\end{Description}
\begin{Usage}
\begin{verbatim}
givens.rotation(theta, p=2, which=c(1, 2))
\end{verbatim}
\end{Usage}
\begin{Arguments}
\begin{ldescription}
\item[\code{theta}] an angle in radians
\item[\code{p}] the dimension of the matrix to be produced
\item[\code{which}] two numbers between 1 and p giving the columns/rows for the
nonzero elements of the result.
\end{ldescription}
\end{Arguments}
\begin{Value}
Returns a p by p matrix z of all zeroes, except z[which,which] has elements
cos(theta), -sin(theta), sin(theta) and cos(theta), in column order.\end{Value}
\begin{Author}\relax
sandy@stat.umn.edu\end{Author}
\begin{References}\relax
Gene H. Golub and Charles F. Van Loan (1989).  Matrix Computations, Second
Edition.  Baltimore:  Johns Hopkins Press, p. 202.\end{References}
\begin{Examples}
\begin{ExampleCode}
 givens.rotation(1,4,c(1,3))
\end{ExampleCode}
\end{Examples}

\Header{ir}{Inverse regression methods}
\keyword{inverse regression, regression}{ir}
\begin{Description}\relax
This function implements inverse regression, including SIR, SAVE and pHd.\end{Description}
\begin{Usage}
\begin{verbatim}
ir(formula, data=list(), subset, weights, na.action=na.omit, method="sir", 
    model=TRUE, contrasts=NULL, ...)
\end{verbatim}
\end{Usage}
\begin{Arguments}
\begin{ldescription}
\item[\code{formula}] a symbolic description of the model to be fit. The details of
the model are the same as for lm. 
\item[\code{data}] an optional data frame containing the variables in the model.
By default the variables are taken from the environment from
which `ir' is called.
\item[\code{subset}] an optional vector specifying a subset of observations to be
used in the fitting process.
\item[\code{weights}] an optional vector of weights to be used where appropriate.
\item[\code{na.action}] a function which indicates what should happen when the data
contain `NA's.  The default is `na.omit,' which will force
calculations on a complete subset of cases.
\item[\code{method}] This character string specifies the method of fitting.  ``sir"
specifies sliced inverse regression and ``save" specifies sliced
average variance estimation.  ``phdy" uses principal hessian
directions using the response as suggested by Li, and ``phdres" 
uses the LS residuals as suggested by Cook. Other methods may be
added
\item[\code{contrasts}] an optional list. See the `contrasts.arg' of
`model.matrix.default'.
\item[\code{...}] additional items that may be required or permitted by some 
methods.
nslices is the number of slices used by sir and save.  numdir 
is the maximum number of directions used.
\end{ldescription}
\end{Arguments}
\begin{Details}\relax
The general regression problem studies \eqn{F(y|x)}{}, the conditional
distribution of a response \eqn{y}{} given a set of predictors \eqn{x}{}.  
This function provides methods for estimating the dimension and central
subspace of a general regression problem.  That is, we want to find a 
\eqn{p \times d}{p by d} matrix \eqn{B}{} such that 
\deqn{F(y|x)=F(y|B'x)}{}  
Both the dimension \eqn{d}{} and the subspace
\eqn{R(B)}{} are unknown.  These methods make few assumptions.  All the methods
available in this function estimate the unknowns by study of the inverse
problem, \eqn{F(x|y)}{}.  In each, a kernel matrix \eqn{M}{} is estimated such
that the column space of \eqn{M}{} should be close to the central subspace.
Eigenanalysis of \eqn{M}{} is then used to estimate the central subspace.
Objects created using this function have appropriate print, summary and plot
methods.\end{Details}
\begin{Value}
Returns a list of items, including
\begin{ldescription}
\item[\code{M}] A matrix that depends on the method of computing.  The column space
of M should be close to the central subspace.
\item[\code{evalues}] The eigenvalues of M (or squared singular values if M is not
symmetric).
\item[\code{evectors}] The eigenvectors of M (or of M'M if M is not square and
symmetric) ordered according to the eigenvalues.
\item[\code{numdir}] The maximum number of directions to be found.  The output
value of numdir may be smaller than the input value.
\item[\code{ols.coef}] Estimated regression coefficients, excluding the intercept,
for the (weighted) LS fit.
\item[\code{ols.fit}] LS fitted values.
\item[\code{slice.info}] output from sir.slice, used by sir and save.
\item[\code{method}] the inverse regression method used.
\end{ldescription}

Other returned values repeat quantities from input.\end{Value}
\begin{Author}\relax
Sanford Weisberg, <sandy@stat.umn.edu>\end{Author}
\begin{References}\relax
The details of these methods are given by R. D. Cook (1998).  Regression
Graphics.  New York:  Wiley.  Equivalent methods are also available in Arc, R.
D. Cook and S. Weisberg (1999).  Applied Regression Including Computing and
Graphics, New York:  Wiley, www.stat.umn.edu/arc.\end{References}
\begin{SeeAlso}\relax
ir.direction,ir.permutation.test,ir.x,ir.y\end{SeeAlso}
\begin{Examples}
\begin{ExampleCode}
attach(ais)  # the Australian athletes data
#fit inverse regression using sir
m1 <- ir(LBM~Wt+Ht+RCC+WCC, method="sir", nslices = 8)
summary(m1)
ir.permutation.test(m1,npermute=100)
plot(m1)

# repeat, using save:

m2 <- update(m1,method="save")
summary(m2)

# repeat, using phd:

m3 <- update(m2, method="phdres")
\end{ExampleCode}
\end{Examples}

\Header{ir.direction}{Inverse regression estimated central subspace}
\keyword{inverse regression, regression}{ir.direction}
\begin{Description}\relax
After fitting an inverse regression, this function returns an 
\eqn{n \times d}{n by d} matrix whose columns span the estimated
central subspace, where \eqn{n}{} is the number of observations.\end{Description}
\begin{Usage}
\begin{verbatim}
ir.direction(object, which=1:object$numdir, norm=F, x=ir.x(object))
\end{verbatim}
\end{Usage}
\begin{Arguments}
\begin{ldescription}
\item[\code{object}] An inverse regression object.
\item[\code{which}] which vectors are wanted.  The default is to return all the
directions.
\item[\code{norm}] should vectors be rescaled to have unit length? Default = F.
\item[\code{x}] 
\end{ldescription}
an \eqn{m \times p}{m by p} matrix of values of the \eqn{p}{}
predictors, ordinarily equal to the predictors used for estimation.\end{Arguments}
\begin{Details}\relax
Inverse regression method produce a matrix of eigenvectors, say \eqn{v}{}.  This
method returns columns of the matrix product \eqn{xv}{} (possibly scaled so the 
columns have unit length).  If the central subspace has dimension \eqn{d}{},
then the first \eqn{d}{} columns of \eqn{xv}{} span the estimated central
subspace.\end{Details}
\begin{Value}
Returns a matrix.\end{Value}
\begin{Author}\relax
Sanford Weisberg, sandy@stat.umn.edu\end{Author}
\begin{SeeAlso}\relax
ir\end{SeeAlso}

\Header{ir.permutation.test}{Inverse Regression Permutation Tests}
\keyword{inverse regression, regression}{ir.permutation.test}
\begin{Description}\relax
This function computes a permutation test for dimension for any inverse 
regression fitting method.\end{Description}
\begin{Usage}
\begin{verbatim}
ir.permutation.test(object, npermute=50, numdir=object$numdir)
\end{verbatim}
\end{Usage}
\begin{Arguments}
\begin{ldescription}
\item[\code{object}] an inverse regression object created by ir
\item[\code{npermute}] number of permutations to compute, default is 50
\item[\code{numdir}] maximum permitted value of the dimension, with the default from
the object
\end{ldescription}
\end{Arguments}
\begin{Value}
Returns an object of type 'ir.permutation.test' that can be printed or
summarized to give the summary of the test.\end{Value}
\begin{Author}\relax
Sanford Weisberg, sandy@stat.umn.edu\end{Author}
\begin{References}\relax
See www.stat.umn.edu/arc/addons.html, and then select the article
on inverse regression.\end{References}
\begin{SeeAlso}\relax
\code{\Link{ir}}\end{SeeAlso}
\begin{Examples}
\begin{ExampleCode}
attach(ais)  # the Australian athletes data
#fit inverse regression using sir
m1 <- ir(LBM~Wt+Ht+RCC+WCC, method="sir", nslices = 8)
summary(m1)
ir.permutation.test(m1,npermute=100)
\end{ExampleCode}
\end{Examples}

\Header{ir.x}{Inverse regression term matrix}
\alias{ir.y}{ir.x}
\keyword{inverse regression, regression}{ir.x}
\begin{Description}\relax
ir.x returns the matrix of data constructed from the formula for an
inverse regression.  ir.y returns the response.\end{Description}
\begin{Usage}
\begin{verbatim}
ir.x(object)
ir.y(object)
\end{verbatim}
\end{Usage}
\begin{Arguments}
\begin{ldescription}
\item[\code{object}] An inverse regression object
\end{ldescription}
\end{Arguments}
\begin{Value}
ir.x returns an \eqn{n \times p}{n by p} matrix of terms, excluding the 
intercept, constructed from the inverse regression object.  ir.y returns the
response.\end{Value}
\begin{Author}\relax
Sanford Weisberg, <sandy@stat.umn.edu>\end{Author}
\begin{SeeAlso}\relax
ir\end{SeeAlso}
\begin{Examples}
\begin{ExampleCode}
attach(ais)
m1 <- ir(LBM~Ht+Wt+RCC+WCC)
ir.x(m1)
ir.y(m1)
\end{ExampleCode}
\end{Examples}

\Header{markby}{Produce a vector of colors/symbols to mark points}
\keyword{~keyword}{markby}
\begin{Description}\relax
This function creates marking of points by color or symbol for use in graphs.\end{Description}
\begin{Usage}
\begin{verbatim}
markby(z, use="color", values=NULL)
\end{verbatim}
\end{Usage}
\begin{Arguments}
\begin{ldescription}
\item[\code{z}] a variable with a few distict values that will define the groups for
marking.
\item[\code{use}] if equal to \code{"color"}, will returna list of colors.  If
anything else, it will return a list of symbols for marking.
\item[\code{values}] a list with as many values as unique values of z that determine
the colors or symbols.  If this is not set, then the function rainbow is used
for colors and 1:length(unique(z)) is used for symbols.
\end{ldescription}
\end{Arguments}
\begin{Value}
Returns length(z) values that specify the color or symbol for each point.\end{Value}
\begin{Author}\relax
Sanford Weisberg, <sandy@stat.umn.edu>\end{Author}
\begin{References}\relax
This function is to help users familiar with Arc, as discussed in
R. D. Cook and S. Weisberg (1999).  \emph{Applied Regression Including
Computing and Graphics}, New York:  Wiley.\end{References}
\begin{Examples}
\begin{ExampleCode}
  x <- rnorm(100)
  y <- rnorm(100)
  z <- cut(rnorm(100),3)
# Scatterplot, mark using color with groups determined by Status
  plot(x,y,col=markby(z))
# Scatterplot, mark using symbols with groups determined by Status
  plot(x,y,pch=markby(z,use="symbols"))
\end{ExampleCode}
\end{Examples}

\Header{rotplot}{draw many 2D projections of a 3D plot}
\keyword{graphics}{rotplot}
\begin{Description}\relax
This function draws several 2D views of a 3D plot, sort of like a spinning
plot.\end{Description}
\begin{Usage}
\begin{verbatim}
rotplot(x, y, theta=seq(0, pi/2, length = 9), ...)
\end{verbatim}
\end{Usage}
\begin{Arguments}
\begin{ldescription}
\item[\code{x}] a matrix with 2 columns giving the horizontal axes of the full 3D
plot.
\item[\code{y}] the vertical axis of the 3D plot.
\item[\code{theta}] a list of rotation angles
\item[\code{...}] additional arguments passed to coplot
\end{ldescription}
\end{Arguments}
\begin{Details}\relax
For each value of theta, draw the plot of cos(theta)*x[,1]+sin(theta)*x[,2]
versus y.\end{Details}
\begin{Value}
returns a graph object.\end{Value}
\begin{Author}\relax
Sanford Weisberg, sandy@stat.umn.edu\end{Author}
\begin{Examples}
\begin{ExampleCode}
 attach(ais)
 m1 <- ir(LMB ~ Ht + Wt + WCC)  
 rotplot(ir.direction(m1,which=1:2),ir.y(m1),col=markby(Sex))
  \end{ExampleCode}
\end{Examples}

\printindex
\end{document}
