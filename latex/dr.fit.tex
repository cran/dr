\Header{dr.fit}{Fit dimension reduction regression}
\methalias{dr.fit.y}{dr.fit}
\begin{Description}\relax
Internal function that estimates the central subspace.  This will not
typically be called directly by the user.  dr.fit.y is the generic name
of a function that returns the response for the particular dimension
reduction regression object.  Typically, this just returns a centered
version of the response, but for phdres it returns OLS residuals.\end{Description}
\begin{Usage}
\begin{verbatim}
dr.fit(object, numdir=4, tol=1e-07, ...)
dr.fit.y(object)
\end{verbatim}
\end{Usage}
\begin{Arguments}
\begin{ldescription}
\item[\code{object}] dimension reduction regression object 
\item[\code{numdir}] maximum number of dimensions to consider 
\item[\code{tol}] tolerance passed to singular value decomposition 
\item[\code{...}] other arguments passed to dr.fit.M 
\end{ldescription}
\end{Arguments}
\begin{Value}
\begin{ldescription}
\item[\code{evectors }] ordered eigenvectors that describe the estimates of the 
dimension reduction subspace
\item[\code{evalues }] ordered eigenvalues
\item[\code{numdir}] number of eigenvalues
\item[\code{raw.evectors}] eigenvectors of the rotated data
\item[\code{decomp}] which decomposition was used?
\end{ldescription}
\end{Value}
\begin{Author}\relax
Sanford Weisberg <sandy@stat.umn.edu>\end{Author}
\begin{SeeAlso}\relax
\code{\Link{dr}}\end{SeeAlso}

