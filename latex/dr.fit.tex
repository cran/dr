\Header{dr.fit}{Fit dimension reduction regression}
\methalias{dr.fit.default}{dr.fit}
\keyword{regression}{dr.fit}
\keyword{internal}{dr.fit}
\begin{Description}\relax
Internal generic function that estimates the central subspace.
\end{Description}
\begin{Usage}
\begin{verbatim}
dr.fit(object, numdir=4, ...)

\end{verbatim}
\end{Usage}
\begin{Arguments}
\begin{ldescription}
\item[\code{object}] dimension reduction regression object 
\item[\code{numdir}] maximum number of dimensions to consider 
\item[\code{tol}] tolerance passed to singular value decomposition 
\item[\code{...}] other arguments passed to dr.fit.M 
\end{ldescription}
\end{Arguments}
\begin{Details}\relax
This will not
typically be called directly by the user.  At present, the same dr.fit method works 
for all dimension reduction methods implemented in this package, but one could
potentially write a special dr.fit method if needed.

The general outline of this method is as follows.  (1)  A matrix M is computed
by a call to dr.fit.M(object,...), such that the columns of M are estimated to 
fall in the subspace of interest (either the central subspace or the central mean
subspace). (2)  If M is square, its eigenvalues and eigenvectors are computed; if
M is not square, the eigenvalues of M'M are computed. (3) M was computed with scaled
and centered predictors.  The eigenvectors are backtransformed to the original
scale.
\end{Details}
\begin{Value}
\begin{ldescription}
\item[\code{evectors }] ordered eigenvectors that describe the estimates of the 
dimension reduction subspace
\item[\code{evalues }] ordered eigenvalues
\item[\code{numdir}] number of eigenvalues
\item[\code{raw.evectors}] eigenvectors of the rotated data
\item[\code{M}] The kernel matrix.
\end{ldescription}
\end{Value}
\begin{Author}\relax
Sanford Weisberg <sandy@stat.umn.edu>
\end{Author}
\begin{SeeAlso}\relax
\code{\Link{dr}}
\end{SeeAlso}

