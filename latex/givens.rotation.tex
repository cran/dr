\Header{givens.rotation}{Create givens rotation matrix}
\keyword{~keyword}{givens.rotation}
\begin{Description}\relax
For a given angle theta, returns a p by p Givens rotation matrix.\end{Description}
\begin{Usage}
\begin{verbatim}
givens.rotation(theta, p=2, which=c(1, 2))
\end{verbatim}
\end{Usage}
\begin{Arguments}
\begin{ldescription}
\item[\code{theta}] an angle in radians
\item[\code{p}] the dimension of the matrix to be produced
\item[\code{which}] two numbers between 1 and p giving the columns/rows for the
nonzero elements of the result.
\end{ldescription}
\end{Arguments}
\begin{Value}
Returns a p by p matrix z of all zeroes, except z[which,which] has elements
cos(theta), -sin(theta), sin(theta) and cos(theta), in column order.\end{Value}
\begin{Author}\relax
sandy@stat.umn.edu\end{Author}
\begin{References}\relax
Gene H. Golub and Charles F. Van Loan (1989).  Matrix Computations, Second
Edition.  Baltimore:  Johns Hopkins Press, p. 202.\end{References}
\begin{Examples}
\begin{ExampleCode}
 givens.rotation(1,4,c(1,3))
\end{ExampleCode}
\end{Examples}

