\Header{dr.weights}{Estimate weighting for elliptical symmetry}
\alias{robust.center.scale}{dr.weights}
\keyword{regression}{dr.weights}
\begin{Description}\relax
This function estimate weights to apply to the rows of a data matrix to
make the resulting weighted matrix as close to multivariate normality as
possible.  This method is usually not called directly by the user.
\end{Description}
\begin{Usage}
\begin{verbatim}
(formula, data = list(), subset, na.action = na.fail, 
    covmethod = "mve", nsamples=NULL, ...) 

robust.center.scale(x, method,... )
\end{verbatim}
\end{Usage}
\begin{Arguments}
\begin{ldescription}
\item[\code{formula}] A one-sided or two-sided formula.  The right hand side is used
to define the design matrix.
\item[\code{data}] An optional data frame.
\item[\code{subset}] A list of cases to be used in computing the weights.
\item[\code{na.action}] The default is na.fail, to prohibit computations.  If set to
na.omit, the function will return a list of weights of the wrong length for use
with dr.
\item[\code{covmethod}] covmethod is passed as the argument \code{method} 
to the function \code{cov.rob} in the required package
\code{lqs}.  The choices are \code{"classical"},
\code{"mve"} and \code{"mcd"}.  Arguments are different with S-Plus.
If classical is selected, the usual estimate of the covariance matrix is
used, but the center is the medians, not the means.
\item[\code{nsamples}] The weights are determined by random sampling from a
data-determined normal distribution.  This controls the number of samples.  The
default is 10 times the number of cases.
\item[\code{x}] An \eqn{n \times p}{n by p} data matrix with no missing values
\item[\code{method}] see covmethod above
\item[\code{...}] Additional args are passed to \code{cov.rob}
\end{ldescription}
\end{Arguments}
\begin{Details}\relax
The basic outline is:  (1) Estimate a mean m and covariance matrix S using a
possibly robust method; (2) For each iteration, obtain a random vector
from N(m,sigma*S).  Add 1 to a counter for observation i if the i-th row
of the data matrix is closest to the random vector; (3) return as weights
the sample faction allocated to each observation.  If you set the keyword
\code{weights.only} to \code{T} on the call to \code{dr}, then only the
list of weights will be returned.
\end{Details}
\begin{Value}
Returns a list of n weights, some of which may be zero.
\end{Value}
\begin{Author}\relax
Sanford Weisberg, sandy@stat.umn.edu
\end{Author}
\begin{References}\relax
R. D. Cook and C. Nachtsheim (1994), Reweighting to achieve
elliptically contoured predictors in regression.  Journal of the American
Statistical Association, 89, 592--599.
\end{References}
\begin{SeeAlso}\relax
\code{\Link{dr}}, \code{\Link{lqs}}
\end{SeeAlso}

