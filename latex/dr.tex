\Header{dr}{Dimension reduction regression}
\methalias{dr.compute}{dr}
\alias{plot.dr}{dr}
\alias{print.dr}{dr}
\alias{print.summary.dr}{dr}
\alias{summary.dr}{dr}
\keyword{regression}{dr}
\begin{Description}\relax
The function dr implements dimension reduction methods, including SIR, SAVE and pHd.
'dr' calls 'dr.compute', so only the former will be needed by most users.
\end{Description}
\begin{Usage}
\begin{verbatim}
dr (formula, data, subset, na.action = na.fail, weights, 
    ...)
    
dr.compute (x, y, weights, method = "sir", ...)
 \end{verbatim}
\end{Usage}
\begin{Arguments}
\begin{ldescription}
\item[\code{formula}] a symbolic description of the model to be fit. The details of
the model are the same as for lm. Although factors may not be 
appropriate for dr methods, they are permitted.  Full rank models are
recommended, although rank deficient models are permitted.
\item[\code{data}] an optional data frame containing the variables in the model.
By default the variables are taken from the environment from
which `dr' is called.
\item[\code{subset}] an optional vector specifying a subset of observations to be
used in the fitting process.
\item[\code{weights}] an optional vector of weights to be used where appropriate.  In the
context of dimension reduction methods, weights are used to obtain
elliptical symmetry, not constant variance; see \code{\Link{dr.weights}}.
\item[\code{na.action}] a function which indicates what should happen when the data
contain `NA's.  The default is `na.fail,' which will stop calculations.
The option 'na.omit' is also permitted, but it may not work correctly when
weights are used.
\item[\code{x}] The design matrix
\item[\code{y}] The response vector or matrix
\item[\code{method}] This character string specifies the method of fitting.  ``sir"
specifies sliced inverse regression and ``save" specifies sliced
average variance estimation.  ``phdy" uses principal hessian
directions using the response as suggested by Li, and ``phdres" 
uses the LS residuals as suggested by Cook. Other methods may be
added.
\item[\code{...}] For 'dr', arguments passed to 'dr.compute'.  For 'dr.compute', 
arguments required for particular dimension reduction method.
\code{nslices} is the number of slices used by sir and save.
\code{numdir} is the maximum number of directions to compute, with
default equal to 4. other methods may have other defaults.
\end{ldescription}
\end{Arguments}
\begin{Details}\relax
The general regression problem studies \eqn{F(y|x)}{}, the conditional
distribution of a response \eqn{y}{} given a set of predictors \eqn{x}{}.  
This function provides methods for estimating the dimension and central
subspace of a general regression problem.  That is, we want to find a 
\eqn{p \times d}{p by d} matrix \eqn{B}{} such that 
\deqn{F(y|x)=F(y|B'x)}{}  
Both the dimension \eqn{d}{} and the subspace
\eqn{R(B)}{} are unknown.  These methods make few assumptions.  All the methods
available in this function estimate the unknowns by study of the inverse
problem, \eqn{F(x|y)}{}.  In each, a kernel matrix \eqn{M}{} is estimated such
that the column space of \eqn{M}{} should be close to the central subspace.
Eigenanalysis of \eqn{M}{} is then used to estimate the central subspace.
Objects created using this function have appropriate print, summary and plot
methods.

Weights can be used, essentially to specify the relative 
frequency of each case in the data.  Empirical weights that make 
the contours of the weighted sample closer to elliptical can be 
computed using \code{\Link{dr.weights}}.  
This will usually result in zero weight for some 
cases.  The function will set zero estimated weights to missing.

Several functions are provided that require a dr object as input.  
\code{dr.permutation.tests} uses a permutation test to obtain significance levels
for tests of dimension.  \code{dr.coplot} allows visualizing the results using a
coplot of either two selected directions conditioning on a third and using
color to mark the response, or the resonse versus one direction,
conditioning on a second direction.  \code{plot.dr} provides the default plot
method for dr objects, based on a scatterplot matrix.
\end{Details}
\begin{Value}
dr returns an object that inherits from dr (the name of the type is the
value of the \code{method} argument), with attributes:
\begin{ldescription}
\item[\code{x}] The design matrix
\item[\code{y}] The response vector
\item[\code{weights}] The weights used, normalized to add to n.
\item[\code{qr}] QR factorization of x.
\item[\code{cases}] Number of cases used.
\item[\code{call}] The initial call to 'dr'.
\item[\code{M}] A matrix that depends on the method of computing.  The column space
of M should be close to the central subspace.
\item[\code{evalues}] The eigenvalues of M (or squared singular values if M is not
symmetric).
\item[\code{evectors}] The eigenvectors of M (or of M'M if M is not square and
symmetric) ordered according to the eigenvalues.
\item[\code{numdir}] The maximum number of directions to be found.  The output
value of numdir may be smaller than the input value.
\item[\code{slice.info}] output from 'sir.slice', used by sir and save.
\item[\code{method}] the dimension reduction method used.
\end{ldescription}
 

\code{dr.weights} returns a vector of weights estimated weights, scaled to add to
the number of cases.
\end{Value}
\begin{Author}\relax
Sanford Weisberg, 
sandy@stat.umn.edu 

For weights, see
R. D. Cook and C. Nachtsheim (1994), Reweighting to achieve
elliptically contoured predictors in regression.  Journal of the American
Statistical Association, 89, 592--599.
\end{Author}
\begin{References}\relax
The details of these methods are given by R. D. Cook 
(1998).  Regression Graphics.  New York:  Wiley.  Equivalent 
methods are also available in Arc, R. D. Cook and S. Weisberg 
(1999).  Applied Regression Including Computing and Graphics, New 
York:  Wiley, www.stat.umn.edu/arc.
\end{References}
\begin{SeeAlso}\relax
\code{\Link{dr.permutation.test}},\code{\Link{dr.x}},\code{\Link{dr.y}},
\code{\Link{dr.direction}},\code{\Link{dr.coplot}},\code{\Link{dr.weights}}
\end{SeeAlso}
\begin{Examples}
\begin{ExampleCode}
library(dr)
data(ais)
attach(ais)  # the Australian athletes data
#fit dimension reduction using sir
m1 <- dr(LBM~Wt+Ht+RCC+WCC, method="sir", nslices = 8)
summary(m1)

# repeat, using save:

m2 <- update(m1,method="save")
summary(m2)

# repeat, using phd:

m3 <- update(m2, method="phdres")
summary(m3)
\end{ExampleCode}
\end{Examples}

