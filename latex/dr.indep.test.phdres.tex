\Header{dr.indep.test.phdres}{Test of independence for pHd based on residuals}
\alias{cov.ew.matrix}{dr.indep.test.phdres}
\begin{Description}\relax
Compute the p-value for the hypothesis that the central
subspace for OLS residuals has dimension 0 against the alternative that it
is greater than zero.  This is an
internal method not usually needed by the user.\end{Description}
\begin{Usage}
\begin{verbatim}
dr.indep.test.phdres(object, stat)
\end{verbatim}
\end{Usage}
\begin{Arguments}
\begin{ldescription}
\item[\code{object}] phdres object 
\item[\code{stat}] value of the test statistic 
\end{ldescription}
\end{Arguments}
\begin{Author}\relax
Sanford Weisberg <sandy@stat.umn.edu>\end{Author}
\begin{References}\relax
R. D. Cook (1998) Regression Graphics, New York:  Wiley, p. xxx.\end{References}
\begin{SeeAlso}\relax
\code{\Link{dr}}\end{SeeAlso}
\begin{Examples}
\begin{ExampleCode}
##---- Should be DIRECTLY executable !! ----
##-- ==>  Define data, use random,
##--         or do  help(data=index)  for the standard data sets.

The function is currently defined as
function(object,stat) {
  eval <- eigen(cov.ew.matrix(object,scaled=F),only.values=T)
  pval<-wood.pvalue(.5*eval$values,stat)
# report results
    z<-data.frame(cbind(stat,pval))
    dimnames(z)<-list(c("Test of independence"),c("Stat","p-value"))
    z}
\end{ExampleCode}
\end{Examples}

