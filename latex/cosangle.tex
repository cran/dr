\Header{cosangle}{Compute the cosine of the angle between a vector and a subspace}
\alias{cosangle1}{cosangle}
\keyword{regression}{cosangle}
\begin{Description}\relax
\code{cosangle1} returns the cosine of the
angle between a vector vecs  and the subspace spanned
by the columns of the matrix mat.  For cosangle, vecs can be a matrix, in
which case cosangle1 is called for each column in vecs.
\end{Description}
\begin{Usage}
\begin{verbatim}
cosangle(mat, vecs)
cosangle1(mat, vec)
\end{verbatim}
\end{Usage}
\begin{Arguments}
\begin{ldescription}
\item[\code{mat}] A matrix the provides the basis for a subspace
\item[\code{vecs}] A vector with the same number of rows as mat, or a matrix
with the same number of rows as mat
\item[\code{vec}] A vector with the same number of rows as mat
\end{ldescription}
\end{Arguments}
\begin{Details}\relax
\code{cosangle1} computes the cosine of the angle between vec and the orthogonal projection
of vec onto the column space of mat.  \code{cosangle} repeats this computation
for each column of vecs.
\end{Details}
\begin{Value}
cosangle1 returns a single value and cosangle returns a vector.
\end{Value}
\begin{Author}\relax
Sanford Weisberg <sandy@stat.umn.edu>
\end{Author}

