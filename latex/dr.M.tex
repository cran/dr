\Header{dr.M}{~~function to do ... ~~}
\methalias{dr.M.mphd}{dr.M}
\methalias{dr.M.mphdq}{dr.M}
\methalias{dr.M.mphdres}{dr.M}
\methalias{dr.M.msir}{dr.M}
\methalias{dr.M.ols}{dr.M}
\methalias{dr.M.phd}{dr.M}
\methalias{dr.M.phdq}{dr.M}
\methalias{dr.M.phdres}{dr.M}
\methalias{dr.M.phdy}{dr.M}
\methalias{dr.M.save}{dr.M}
\methalias{dr.M.sir}{dr.M}
\keyword{internal}{dr.M}
\begin{Description}\relax
Compute the kernel matrix for a dr method
\end{Description}
\begin{Usage}
\begin{verbatim}
dr.M(object, ...)
\end{verbatim}
\end{Usage}
\begin{Arguments}
\begin{ldescription}
\item[\code{object}] An object of type 'dr'. 
\item[\code{...}] Additional method-specific arguments; see description below
\end{ldescription}
\end{Arguments}
\begin{Details}\relax
This is a method-specific function to compute the kernel matrix M of a
dimension reduction method.  For sir, save, msir and save, the additional 
arguements are 'nslices', the number of slices to be used, and 'slice.info',
which if present is the a list produced by \code{\Link{dr.slices}}.  For ols and
the phd methods, no additional arguments are required.

New dimension reduction methods can be added to 'dr' by writing a new 'dr.M' 
method.
\end{Details}
\begin{Value}
Returns a list with components
\begin{ldescription}
\item[\code{M }] The kernel matrix
\item[\code{slice.info }] Information about the slices, if used by the dr method.
\end{ldescription}

...
\end{Value}
\begin{Author}\relax
Sanford Weisberg, sandy@stat.umn.edu
\end{Author}
\begin{SeeAlso}\relax
~~objects to See Also as \code{\Link{dr}}, ~~~
\end{SeeAlso}
\begin{Examples}
\begin{ExampleCode}
##---- Should be DIRECTLY executable !! ----
##-- ==>  Define data, use random,
##--    or do  help(data=index)  for the standard data sets.

## The function is currently defined as
function(object, ...){UseMethod("dr.M")}
\end{ExampleCode}
\end{Examples}

