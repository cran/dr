\Header{dr.permutation.test}{Inverse Regression Permutation Tests}
\keyword{inverse regression, regression}{dr.permutation.test}
\begin{Description}\relax
This function computes a permutation test for dimension for any inverse 
regression fitting method.\end{Description}
\begin{Usage}
\begin{verbatim}
dr.permutation.test(object, npermute=50, numdir=object$numdir)
\end{verbatim}
\end{Usage}
\begin{Arguments}
\begin{ldescription}
\item[\code{object}] an inverse regression object created by dr
\item[\code{npermute}] number of permutations to compute, default is 50
\item[\code{numdir}] maximum permitted value of the dimension, with the default from
the object
\end{ldescription}
\end{Arguments}
\begin{Value}
Returns an object of type 'dr.permutation.test' that can be printed or
summarized to give the summary of the test.\end{Value}
\begin{Author}\relax
Sanford Weisberg, sandy@stat.umn.edu\end{Author}
\begin{References}\relax
See www.stat.umn.edu/arc/addons.html, and then select the article
on inverse regression.\end{References}
\begin{SeeAlso}\relax
\code{\Link{dr}}\end{SeeAlso}
\begin{Examples}
\begin{ExampleCode}
data(ais)
attach(ais)  # the Australian athletes data
#fit inverse regression using sir
m1 <- dr(LBM~Wt+Ht+RCC+WCC, method="sir", nslices = 8)
summary(m1)
dr.permutation.test(m1,npermute=100)
\end{ExampleCode}
\end{Examples}

