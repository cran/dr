\Header{dr.permutation.test}{Dimension Reduction Regression Functions}
\methalias{dr.permutation.test.statistic}{dr.permutation.test}
\alias{dr.permutation.test.statistic.default}{dr.permutation.test}
\alias{dr.permutation.test.statistic.phd}{dr.permutation.test}
\alias{dr.permutation.test.statistic.phdres}{dr.permutation.test}
\alias{dr.permutation.test.statistic.phdy}{dr.permutation.test}
\alias{print.dr.permutation.test}{dr.permutation.test}
\alias{summary.dr.permutation.test}{dr.permutation.test}
\keyword{regression}{dr.permutation.test}
\begin{Description}\relax
These functions require a dimension reduction regression object as input to
produce output of various types.
\end{Description}
\begin{Usage}
\begin{verbatim}
dr.permutation.test(object, npermute=50,numdir=object$numdir,permute.weights=TRUE)

\end{verbatim}
\end{Usage}
\begin{Arguments}
\begin{ldescription}
\item[\code{object}] a dimension reduction regression object created by dr
\item[\code{npermute}] number of permutations to compute, default is 50
\item[\code{numdir}] maximum permitted value of the dimension, with the default from
the object
\item[\code{permute.weights}] If TRUE, permute weights as well as data in the
permutation test.  If FALSE, do not permute the weights.
\end{ldescription}
\end{Arguments}
\begin{Details}\relax
Approximates significance levels for tests of dimension using a permutation test.
\end{Details}
\begin{Value}
Returns an object of type 'dr.permutation.test' that can be printed or
summarized to give the summary of the test.
\end{Value}
\begin{Author}\relax
Sanford Weisberg, sandy@stat.umn.edu
\end{Author}
\begin{References}\relax
See www.stat.umn.edu/arc/addons.html, and then select the article
on dimension reduction regression or inverse regression.
\end{References}
\begin{SeeAlso}\relax
\code{\Link{dr}}
\end{SeeAlso}
\begin{Examples}
\begin{ExampleCode}
data(ais)
attach(ais)  # the Australian athletes data
#fit dimension reduction regression using sir
m1 <- dr(LBM~Wt+Ht+RCC+WCC, method="sir", nslices = 8)
summary(m1)
dr.permutation.test(m1,npermute=100)
plot(m1)
dr.coplot(m1)
\end{ExampleCode}
\end{Examples}

