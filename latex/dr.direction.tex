\Header{dr.direction}{Dimension reduction regression estimated central subspace}
\keyword{Dimension reduction regression, regression}{dr.direction}
\begin{Description}\relax
After fitting a dimension reduction regression, this function returns an 
\eqn{n \times d}{n by d} matrix whose columns span the estimated
central subspace, where \eqn{n}{} is the number of observations.\end{Description}
\begin{Usage}
\begin{verbatim}
dr.direction(object, which=1:object$numdir, norm=F, x=dr.x(object))
\end{verbatim}
\end{Usage}
\begin{Arguments}
\begin{ldescription}
\item[\code{object}] An dimension reduction regression object.
\item[\code{which}] which vectors are wanted.  The default is to return all the
directions.
\item[\code{norm}] should vectors be rescaled to have unit length? Default = F.
\item[\code{x}] 
\end{ldescription}
an \eqn{m \times p}{m by p} matrix of values of the \eqn{p}{}
predictors, ordinarily equal to the predictors used for estimation.\end{Arguments}
\begin{Details}\relax
Inverse regression method produce a matrix of eigenvectors, say \eqn{v}{}.  This
method returns columns of the matrix product \eqn{xv}{} (possibly scaled so the 
columns have unit length).  If the central subspace has dimension \eqn{d}{},
then the first \eqn{d}{} columns of \eqn{xv}{} span the estimated central
subspace.\end{Details}
\begin{Value}
Returns a matrix.\end{Value}
\begin{Author}\relax
Sanford Weisberg, sandy@stat.umn.edu\end{Author}
\begin{SeeAlso}\relax
dr\end{SeeAlso}
\begin{Examples}
\begin{ExampleCode}
data(ais)
attach(ais)
m1<-dr(LBM~Wt+Ht+WCC+RCC)
dr.direction(m1,1:2) # spanning vectors for a central subspace of dimension 2.
\end{ExampleCode}
\end{Examples}

