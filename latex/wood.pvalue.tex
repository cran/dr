\Header{wood.pvalue}{Chi-square approximation to a sum of Chi-squares}
\keyword{regression}{wood.pvalue}
\begin{Description}\relax
Returns an approximation to P(coef'X > f) for X=(X1,...,Xk)', a vector of iid
one df chi-squared rvs.  coef is a list of positive coefficients. tol is used
to check for near-zero conditions.
\end{Description}
\begin{Usage}
\begin{verbatim}
wood.pvalue(coef, f, tol=0, print=FALSE)
\end{verbatim}
\end{Usage}
\begin{Arguments}
\begin{ldescription}
\item[\code{coef}] Vector of positive numbers, usually from the eigenvalues of a
random matrix
\item[\code{f}] observed value of statistic
\item[\code{tol}] A small number for tolerance
\item[\code{print}] If TRUE, print the output; if FALSE return the pvalue
\end{ldescription}
\end{Arguments}
\begin{Details}\relax
Following Wood (1989), use either a Saterthwaite-Welsh, inverse gamma or
three-parameter Pearson Type IV approximation, depending on the values of
the coefficients.
\end{Details}
\begin{Value}
Returns the tail area probability.
\end{Value}
\begin{Author}\relax
Sanford Weisberg <sandy@stat.umn.edu>.  Arc version by Dennis
Cook.
\end{Author}
\begin{References}\relax
See Wood (1989), Communications in Statistics, Simulation 1439-1456.
Translated from Arc function wood-pvalue.
\end{References}
\begin{SeeAlso}\relax
\code{\Link{dr}}
\end{SeeAlso}

