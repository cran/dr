\Header{dr.x}{Accessor functions for dr. objects}
\alias{dr.Q}{dr.x}
\alias{dr.qr}{dr.x}
\alias{dr.R}{dr.x}
\alias{dr.wts}{dr.x}
\alias{dr.y.name}{dr.x}
\alias{dr.z}{dr.x}
\keyword{internal}{dr.x}
\begin{Description}\relax
~~ A concise (1-5 lines) description of what the function does. ~~
\end{Description}
\begin{Usage}
\begin{verbatim}
dr.x(object)
dr.wts(object) 
dr.qr(object)
dr.Q(object)
dr.R(object)
dr.z(object) 
dr.y.name(object) 
\end{verbatim}
\end{Usage}
\begin{Arguments}
\begin{ldescription}
\item[\code{object}] An object that inherits from 'dr'. 
\end{ldescription}
\end{Arguments}
\begin{Value}
Returns a component of a dr object.  For example, 'dr.qr' returns object\$qr,
'dr.wts' returns object\$weights, 'dr.Q' returns the Q component of object\$qr,
'dr.z' returns the scaled and centered design matrix.
\end{Value}
\begin{Author}\relax
Sanford Weisberg, sandy@stat.umn.edu
\end{Author}
\begin{SeeAlso}\relax
~~objects to See Also as \code{\Link{dr}}.
\end{SeeAlso}
\begin{Examples}
\begin{ExampleCode}
##---- Should be DIRECTLY executable !! ----
##-- ==>  Define data, use random,
##--    or do  help(data=index)  for the standard data sets.

## The function is currently defined as
function(object) {object$x}
\end{ExampleCode}
\end{Examples}

