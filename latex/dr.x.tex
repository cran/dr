\Header{dr.x}{Inverse regression term matrix}
\alias{dr.y}{dr.x}
\alias{dr.z}{dr.x}
\keyword{Dimension reduction regression, regression}{dr.x}
\begin{Description}\relax
dr.x returns the matrix of data constructed from the formula for a
dimension reduction regression.  dr.y returns the response.\end{Description}
\begin{Usage}
\begin{verbatim}
dr.x(object)
dr.y(object)
dr.z(x,weights=NULL,center=TRUE,rotate=TRUE,decomp="svd")
\end{verbatim}
\end{Usage}
\begin{Arguments}
\begin{ldescription}
\item[\code{object}] An dimension reduction regression object
\item[\code{x}] 
\item[\code{weights}] 
\item[\code{center}] 
\item[\code{rotate}] 
\item[\code{decomp}] 
\end{ldescription}
 Decomposition to be used in computing the rotation; the
default is "svd".\end{Arguments}
\begin{Value}
dr.x returns an \eqn{n \times p}{n by p} matrix of terms, excluding the 
intercept, constructed from the dimension reduction regression object.  dr.y returns the
response. dr.z returns a possibly centered and scaled version of x.\end{Value}
\begin{Author}\relax
Sanford Weisberg, <sandy@stat.umn.edu>\end{Author}
\begin{SeeAlso}\relax
dr\end{SeeAlso}
\begin{Examples}
\begin{ExampleCode}
data(ais)
attach(ais)
m1 <- dr(LBM~Ht+Wt+RCC+WCC)
dr.x(m1)
dr.y(m1)
\end{ExampleCode}
\end{Examples}

