\Header{plot.dr}{Plotting methods for dimension reduction regression}
\alias{coplot.dr}{plot.dr}
\keyword{regression}{plot.dr}
\begin{Description}\relax
These routines provide default plotting methods for dimension reduction regression.\end{Description}
\begin{Usage}
\begin{verbatim}
plot.dr(object, which=1:object$numdir, mark.by.y=F,plot.method=pairs, ...)
dr.coplot(object, which=1:object$numdir, mark.by.y=F, ...)
\end{verbatim}
\end{Usage}
\begin{Arguments}
\begin{ldescription}
\item[\code{object}] Any dimension reduction regression object
\item[\code{which}] Which directions to plot
\item[\code{mark.by.y}] if TRUE, use the response as a marking variable to color
points; if FALSE, use response in the plot
\item[\code{plot.method}] The default is to use the pairs or coplot method to
draw the plots.  If John Fox's car library is available, you can
substitute scatterplot.matrix for pairs.
\item[\code{...}] arguments passed to plot or coplot.  In particular, if the
car library is available, the argument panel=panel.car will add smoothers
to a coplot.
\end{ldescription}
\end{Arguments}
\begin{Value}
Produces a scatterplot matrix (plot) or coplot (dr.coplot) of the specified
directions in an dimension reduction regression\end{Value}
\begin{Author}\relax
Sanford Weisberg, sandy@stat.umn.edu\end{Author}
\begin{References}\relax
Cook, R. D. and Weisberg, S. (1999).  Applied Regression
Including Computing and Graphics.  New York:  Wiley.\end{References}
\begin{Examples}
\begin{ExampleCode}
data(ais)
attach(ais)
i1<-dr(LBM~Ht+Wt+RCC+WCC)
plot(i1)
dr.coplot(i1,mark.by.y=TRUE)
\end{ExampleCode}
\end{Examples}

