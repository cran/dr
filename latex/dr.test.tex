\Header{dr.test}{Dimension reduction regression tests}
\methalias{dr.test.default}{dr.test}
\methalias{dr.test.phd}{dr.test}
\methalias{dr.test.phdq}{dr.test}
\methalias{dr.test.phdres}{dr.test}
\methalias{dr.test.phdy}{dr.test}
\methalias{dr.test.sir}{dr.test}
\alias{dr.test2.phdres}{dr.test}
\keyword{regression}{dr.test}
\begin{Description}\relax
This method computes tests of dimension in dimension reduction regression.
A separate method is needed for each dimension reduction regression method.
Called by the dr function, and not usually accessed directly by the user.
\end{Description}
\begin{Usage}
\begin{verbatim}
dr.test(object, nd) 
dr.test.default(object, nd)
\end{verbatim}
\end{Usage}
\begin{Arguments}
\begin{ldescription}
\item[\code{object}] dimension reduction regresion object 
\item[\code{nd}] Maximum number of dimensions to test 
\end{ldescription}
\end{Arguments}
\begin{Value}
Returns test statistics and significance levels in a data frame.
\end{Value}
\begin{Author}\relax
Sanford Weisberg, <sandy@stat.umn.edu>
\end{Author}
\begin{References}\relax
R. C. Cook (1998).  Regression Graphics. New York:  Wiley.
\end{References}
\begin{SeeAlso}\relax
\code{\Link{dr}}
\end{SeeAlso}

