\Header{dr.persp}{Produce a perspective plot of dimension reduction regression objects.}
\begin{Description}\relax
This function uses the sm library to draw a perspective 3D plot of two of
the directions obtained by dimension reduction regression versus the response.\end{Description}
\begin{Usage}
\begin{verbatim}
dr.persp(object, which=1:2, h=c(0.1, 0.1), ...)
\end{verbatim}
\end{Usage}
\begin{Arguments}
\begin{ldescription}
\item[\code{object}] a dimension reduction regression object 
\item[\code{which}] a vector of two directions to use 
\item[\code{h}] bandwidths for smoothing passed to sm 
\item[\code{...}] graphical parameters passed to sm 
\end{ldescription}
\end{Arguments}
\begin{Value}
Returns a perspective plot\end{Value}
\begin{Author}\relax
Sanford Weisberg <sandy@stat.umn.edu>\end{Author}
\begin{SeeAlso}\relax
\code{\Link{dr}}, \code{\Link{sm}}\end{SeeAlso}
\begin{Examples}
\begin{ExampleCode}
##---- Should be DIRECTLY executable !! ----
##-- ==>  Define data, use random,
##--         or do  help(data=index)  for the standard data sets.

The function is currently defined as
function(object,which=1:2,h=c(.1,.1),...){
 require(sm) # uses the sm library
 if (length(which) == 2){
  d1<-dr.direction(object,which,norm=T)
  y<-dr.y(object)
  sm.regression(d1,y,h=h,
                     xlab=dimnames(d1)[[2]][1],
                     ylab=dimnames(d1)[[2]][2],
                     zlab=names(object$model[1]))
  }
  else
  print("This method requires specifying two directions")
 }
\end{ExampleCode}
\end{Examples}

