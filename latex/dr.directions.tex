\Header{dr.directions}{Directions selected by dimension reduction regressiosn}
\alias{dr.direction}{dr.directions}
\methalias{dr.direction.default}{dr.directions}
\keyword{regression}{dr.directions}
\begin{Description}\relax
Dimension reduction regression returns a set of p orthogonal direction
vectors each of length p, the first d of which are estimates a basis of a 
d dimensional central subspace.  The function returns the estimated directions 
in the original n dimensional space for plotting.
\end{Description}
\begin{Usage}
\begin{verbatim}
dr.direction(object, which, norm, x)
dr.directions(object, which, norm, x)
dr.direction.default(object, which=1:object$numdir,norm=FALSE,x=dr.x(object))
\end{verbatim}
\end{Usage}
\begin{Arguments}
\begin{ldescription}
\item[\code{object}] a dimension reduction regression object 
\item[\code{which}] select the directions wanted, default is all directions 
\item[\code{norm}] if TRUE, direction vectors are normalized to length 1, otherwise
their length is arbitrary
\item[\code{x}] select the X matrix, the default is dr.x(object)
\item[\code{...}] additional arguments are passed to dr.direction.default
\end{ldescription}
\end{Arguments}
\begin{Details}\relax
Dimension reduction regression is used to estimate a basis of the central
subspace of a regression.  If there are p predictors, the dimension
reduction regression object includes a p by p matrix of C of eigenvectors.
This method returns (X-m1')C where m is the vector of column means of X.  If
X is equal to the original matrix of predictors given by dr.x(object), then
this gives the directions in the coordinates of the orginal n dimensional
space.  These directions are used in graphical methods and elsewhere.
\end{Details}
\begin{Value}
Returns a matrix.  The same function has two names.
\end{Value}
\begin{Author}\relax
Sanford Weisberg <sandy@stat.umn.edu>
\end{Author}
\begin{References}\relax
See R. D. Cook (1998).  Regression Graphics.  New York:  Wiley.
\end{References}
\begin{SeeAlso}\relax
\code{\Link{dr}}
\end{SeeAlso}
\begin{Examples}
\begin{ExampleCode}
library(dr)
data(ais)
attach(ais)  # the Australian athletes data
#fit dimension reduction using sir
m1 <- dr(LBM~Wt+Ht+RCC+WCC, method="sir", nslices = 8)
summary(m1)
dr.directions(m1)
\end{ExampleCode}
\end{Examples}

