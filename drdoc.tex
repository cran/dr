\documentclass{article}
\usepackage[ae]{Rd}
\usepackage{html}
%\usepackage{times}
\newcommand{\Splus}{{\normalfont\textsf{Splus}}{}}
\newcommand{\arc}{{\em Arc}}
\newcommand{\xls}{{\em Xlisp-Stat}}
\newcommand{\dcode}[1]{{\small{\tt #1}}}
\newcommand{\ir}{{\tt dr}}
\newcommand{\dr}{{\tt dr}}
\newcommand{\sir}{{\sffamily sir}}
\newcommand{\save}{{\sffamily save}}
\newcommand{\phd}{{\sffamily phd}}
\newcommand{\phdy}{{\sffamily phdy}}
\newcommand{\phdq}{{\sffamily phdq}}
\newcommand{\phdres}{{\sffamily phdres}}
\renewcommand{\span}{{\mathcal S}}
\newcommand{\E}{\mbox{E}}
\newcommand{\N}{\mbox{N}}
\newcommand{\indep}{\;\, \rule[0em]{.03em}{.6em} \hspace{-.25em}
    \rule[0em]{.65em}{.03em} \hspace{-.25em} \rule[0em]{.03em}{.6em}\;\,}

\newenvironment{references}{%
  \section{References}%
    \begin{list}%
            {}%
            {\setlength{\leftmargin}{.25in}\setlength{\itemindent}{-.25in}}}%
   {\end{list}}

\newcommand{\jasa} {{\it Journal of the American Statistical Association}}
\newcommand{\jrssa}{{\it Journal of the Royal Statistical Society, Series A}}
\newcommand{\jrssb}{{\it Journal of the Royal Statistical Society, Series B}}
\newcommand{\jrssc}{{\it Applied Statistics}}
\newcommand{\techx}{{\it Technometrics}}
\newcommand{\Wiley}{New York: Wiley}


\title{Dimension Reduction Regression in \R}
\author{Sanford Weisberg\thanks{
{\em Journal of Statistical Software}, 2002, Volume 7, available from
http://www.jstatsoft.org}\\
{\small \sl
School of Statistics, University of Minnesota, St. Paul,
MN 55108-6042.}\\
{\small Supported by National Science Foundation Grant DUE 0109756.}}
\date{\today}
\begin{document}
\maketitle

\begin{abstract}
Regression is the study of the dependence of a response variable
$y$ on a collection $p$ predictors collected in $x$. In {\em
dimension reduction regression}, we seek to find a few linear
combinations $\beta_1'x,\ldots,\beta_d'x$, such that all the
information about the regression is contained in these $d$ linear
combinations.  If $d$ is very small, perhaps one or two, then the
regression problem can be summarized using simple graphics; for
example, for $d=1$, the plot of $y$ versus $\beta_1'x$ contains
all the regression information.  When $d=2$, a 3D plot contains
all the information.

Several methods for estimating $d$ and relevant functions of
$\beta_1\ldots,\beta_d$ have been suggested in the literature.  In
this paper, we describe an
\R\ package for three important dimension reduction methods:
sliced inverse regression or
\sir, sliced average variance estimates, or \save, and principal
Hessian directions, or \phd.  The package is very general and
flexible, and can be easily extended to include other methods of
dimension reduction.  It includes tests and estimates of the
dimension $d$, estimates of the relevant information including
$\beta_1,\ldots,\beta_d$, and some useful graphical summaries as
well.
\end{abstract}

\section{Introduction}
In the general regression problem, we have a response $y$ of
dimension $k$ (usually, $k=1$) and $p$-dimensional predictor $x$,
and the goal is to learn about how the conditional distributions
$F(y|x)$ as $x$ varies through its sample space.  In parametric
regression, we specify a functional form for the conditional
distributions that is known up to a few parameters.  In
nonparametric regression, no assumptions are made about $F$, but
progress is really only possible if the dimensions $p$ and $k$ are
small.

{\em Dimension reduction regression} is one intermediate
possibility between the parametric and nonparametric extremes.  In
this setup, we assume without loss of information that the
conditional distributions can be indexed by $d$ linear
combinations, or for some probably unknown $p\times d$ matrix $B$
\begin{equation}
F(y|x) = f(y|B'x)
\label{eq:eq1}
\end{equation}
or in words, all the information in $x$ about $y$ is contained in
the $d$ linear combinations $B'x$.
This representation always holds trivially, by setting $B=I$, the
$p\times p$ identity matrix, and so the usual goal is to find the
$B$ of lowest possible dimension for which this representation
holds.  If (\ref{eq:eq1}) holds for a particular $B$, then it also
holds for $B^*=BA$, where $A$ is any full rank matrix, and hence
the unique part of the regression summary is the subspace that is
spanned by $B$, which we denote $\span(B)$. Cook (1998) provides a
more complete introduction to these ideas, including discussion of
when this subspace, which we call the {\em central subspace},
exists, and when it is unique.

In this paper, we discuss software for estimating the subspace
$\span(B)$ spanned by $B$, and tests concerning the dimension $d$
based on dimension reduction methods. This software was written
using \R, but can also be used with
\Splus.  Most, but not all, of the methods available here are also
included in the
\xls\ program
\arc, Cook and
Weisberg (1999).  The \R\ platform allows use of dimension reduction
methods with existing statistical methods that are not readily available in
\xls, and hence in \arc.  For example, the \R\ functions are more suitable for
Monte Carlo experimentation than \arc.  In addition, \R\ includes a much wider
array of options for smoothing, including multidimensional smoothers.
On the other hand,
\arc\ takes full advantage of the dynamic graphical capabilities of \xls, and at
least for now the graphical summaries of dimension reduction regression are
clearly superior in \arc.  Thus, there appears to be good reason to have
these methods available using both platforms.

Cook (1998) provides the most complete introduction to this area, and we
leave the details of all the methods described here to that book.
See also Cook and Weisberg (1994) for a more gentle introduction
to dimension reduction. In this paper we give only the barest
outline of dimension reduction methodology, concentrating on the
software.


Suppose we have data $(x_i,y_i)$, for $i=1,\ldots,n$ that are
independent and collected into a matrix $X$ and a vector $Y$ if
$k=1$ and a matrix $Y$ if $k>1$.  In addition, suppose we have
nonnegative weights $w_1,\ldots,w_n$ whose sum is $n$;
 if unspecified, we take all the $w_i=1$.  Generally following Yin
(2000), a procedure for estimating $\span(B)$ and for obtaining
tests concerning $d$ is:
\begin{enumerate}
    \item Scale and center $X$ as
    \[
    Z =
    W^{1/2}(X - 1\bar{x}') \hat{\Sigma}^{-1/2}
    \]
    where $\bar{x} = \sum w_ix_i/\sum w_i$ is the vector of weighted column
    means, $W = \mathrm{diag}(w_i)$, and
    \[
    \hat{\Sigma} = \frac{1}{n -1} (X - 1\bar{x}')'W(X - 1\bar{x}')
    \]
    $\hat{\Sigma}^{-1/2}$ is any square root of the inverse of the
    sample covariance matrix for $X$ (for example, using a singular
    value decomposition) and $\bar{x}$ is a vector of weighted
    sample means.  In this
    scaling, the rows of $Z$ have zero mean and identity sample covariance
    matrix.
    \item
    Use the scaled and centered data $Z$ to find a
    dimension $p \times p$ symmetric
    matrix $\hat{M}$ that is a consistent
    estimate of a population matrix $M$ with the property that $\span(M)
    \subseteq \span(B)$.  For most procedures, all we can guarantee is
    that $M$ tells us about a part, but not necessarily all, of
    $\span(B)$.  Each of the methods (for example, \sir, \save, and
    \phd) have a different method for selecting $\hat{M}$.
    \item
    Let $|\hat{\lambda}_1|\geq \ldots \geq |\hat{\lambda}_p|$ be the
    ordered absolute eigenvalues
    of $\hat{M}$, and $\hat{u}_1,\ldots,\hat{u}_p$  the
    corresponding eigenvectors of $\hat{M}$.
    In some applications (like \phd) the eigenvalues may be negative.
    \item
    A test that the dimension $d=d_0$  against the alternative that $d >
    d_0$ is based on a partial sum of eigenvalues of the form:
    \[
    \Lambda_{d_0} = n \hat{c} \sum_{j=d_0+1}^p |\hat{\lambda}_j|^{\nu}
    \]
    where $\hat{c}$ is a method-specific term, and $\nu$ is generally
    equal to 1, but it is equal to 2 for \phd.
    The distribution of these partial sums depends on assumptions and on
    the method of obtaining $\hat{M}$.
    \item
    Given $d$, the estimate of $\span(B)$ is the span of the first
    $d$ eigenvectors.  When viewed as a subspace of $\Re^n$, the basis
    for this estimated subspace is
    $Z\hat{u}_1,\ldots,Z\hat{u}_d$.  These
    directions can then be back-transformed to the $X$-scale.  Given the
    estimate of $\span(B)$, graphical methods can be used to recover
    information about $F$, or about particular aspects of the conditional
    distributions, such as the conditional mean function.
\end{enumerate}

\section{Usage}
The \R\ package for dimension reduction called \dr\ can be
obtained from the Comprehensive R Archive Network, CRAN, at
\verb+http://www.R-project.org+.
The primary function in this package is called \ir.  Its usage is
modeled after the \R\ function \dcode{lm} that is used for fitting
linear models, and so many of the arguments that can be used in
\dcode{lm} can also be used in \ir.  The following statements will
examine dimension reduction using \sir\ applied to the Australian
Institute of Sport data:
\small
\begin{verbatim}
> library(dr)
> data(ais)
> attach(ais)
> i1 <- dr(LBM~ Ht + Wt + log(RCC) + WCC, method="sir", nslices=8)
\end{verbatim}
\normalsize
\noindent First, provide a formula giving the response $Y$ on the
left side, and the predictors on the right side. The response can
be a vector of $n$ elements or for \sir\ and
\save\ it can be a matrix with $n$ rows and $k$ columns.
The model statement on the right side can include factors and
interactions, or transformations, so the syntax is very general.
%As with \dcode{lm}, you can
%also specify a vector for $Y$ followed by a matrix for $X$ in place of the
%model statement.
The remaining arguments are all optional, and the values shown are
all defaults.  \dcode{method} can equal either \dcode{"sir"},
\dcode{"save"},  \dcode{"phd"}, \dcode{"phdres"}, \dcode{"phdq"}
or
\dcode{"phdy"}.
Adding other methods is not
difficult, and is described in Section~\ref{sec:newmeth}.
The number of slices is relevant only for the methods that use slicing,
\sir\ and \save, and has default equal to the larger of 8
and the number of predictors plus 3. If the response has $k$
columns, then the argument \dcode{nslices} should be a vector of
$k$ elements.  If it is specified as a number rather than a
vector, then that number will give the total number of cells,
approximately. For example, if $k=2$ and \dcode{nslices=8}, the
program will slice $\sqrt{8}
\approx 3$ slices along each of the two response variables for a
total of $3\times 3=9$ cells. The argument \dcode{numdir}
specifies the number of directions to display in printed output
and the number of tests to compute.  The default is 4.

Keywords inherited from \dcode{lm} include \dcode{weights},
which if set should be a
vector of the same length of the response of positive numbers;
\dcode{contrasts},
which specifies how to turn factors into variables;
\dcode{na.action}, which specifies how to ``handle" missing
values.  The default for this argument is
\dcode{na.omit}, which will reduce the dataset by eliminating all cases with
one or more missing value.  The argument \dcode{subset} can be set
to a list of case indices to be used in the computations; all
other cases are ignored.

Brief printed output from the \dr\ method is obtained by simply typing
the name of the object created, \dcode{i1} in the above example:
\small\begin{verbatim}
> i1

Call:
dr(formula = LBM ~ Ht + Wt + log(RCC) + WCC, method = "sir", nslices = 8)

Eigenvectors:
                Dir1          Dir2        Dir3         Dir4
Ht        0.01054752 -0.0001569418 -0.10750718  0.009197952
Wt        0.02374812  0.0040912479  0.06248766 -0.019286750
log(RCC)  0.99960915 -0.9999614779  0.99211547  0.757087590
WCC      -0.01031144  0.0077640009 -0.01563290  0.652963858
Eigenvalues:
[1] 0.87789585 0.15017504 0.03972711 0.01737281
\end{verbatim}
\normalsize
\noindent This output repeats the call that created the object, and basic summary
statistics, described more completely in Section~\ref{sec:sir}.  Using the
\dcode{summary} method gives more complete output:
\small\begin{verbatim}
> summary(i1)

Call:
dr(formula = LBM ~ Ht + Wt + log(RCC) + WCC, method = "sir", nslices = 8)

Terms:
LBM ~ Ht + Wt + log(RCC) + WCC

Method:
sir with 8 slices, n = 202.

Slice Sizes:
26 26 25 25 25 27 30 18

Eigenvectors:
             Dir1       Dir2     Dir3      Dir4
Ht        0.01055  0.0001569 -0.10751  0.009198
Wt        0.02375 -0.0040912  0.06249 -0.019287
log(RCC)  0.99961  0.9999615  0.99212  0.757088
WCC      -0.01031 -0.0077640 -0.01563  0.652964

              Dir1   Dir2    Dir3    Dir4
Eigenvalues 0.8779 0.1502 0.03973 0.01737
R^2(OLS|dr) 0.9986 0.9987 0.99978 1.00000

Asymp. Chi-square tests for dimension:
               Stat df  p-value
0D vs >= 1D 219.205 28 0.000000
1D vs >= 2D  41.870 18 0.001153
2D vs >= 3D  11.534 10 0.317440
3D vs >= 4D   3.509  4 0.476465
\end{verbatim}
\normalsize
\noindent
This output repeats most of the basic output, plus it also gives information
on the slices and on tests of dimension, if available.

The class of the object created by \dr\ depends on the value of
the argument \dcode{method}.  For example, if
\dcode{method="sir"}, the object is of class \dcode{sir} if the response is
univariate and \dcode{msir} if the response is multivariate.  If
\dcode{method="save"}, the object is of class \dcode{save} or \dcode{msave}.
All these objects inherit from the class \dcode{dr}.

Several additional quantities are computed and stored in the object.  These
include:
\small
\begin{verbatim}
> names(i1)
 [1] "formula"          "contrasts"        "xlevels"          "call"
 [5] "ols.coef"         "ols.fit"          "weights"          "cols.used"
 [9] "offset"           "estimate.weights" "terms"            "method"
[13] "response.name"    "model"            "cases"            "evectors"
[17] "evalues"          "numdir"           "raw.evectors"     "decomp"
[21] "M"                "slice.info"
\end{verbatim}
\normalsize
\noindent
For example, to have access to the eigenvectors of $\hat{M}$, type
\small\begin{verbatim}
> i1$evectors
                Dir1          Dir2        Dir3         Dir4
Ht        0.01054752  0.0001569418 -0.10750718  0.009197952
Wt        0.02374812 -0.0040912479  0.06248766 -0.019286750
log(RCC)  0.99960915  0.9999614779  0.99211547  0.757087590
WCC      -0.01031144 -0.0077640009 -0.01563290  0.652963858
\end{verbatim}
\normalsize
\noindent while {\small \verb+i1$M+} %$
returns the value of the matrix $\hat{M}$.  \dcode{ols.fit} returns ols
fitted values (or weighted least squares fitted values, if appropriate)
which are computed and used by some of the methods and summaries.

\section{Methods available}
\subsection{Sliced inverse regression}\label{sec:sir}
Sliced inverse regression, or \sir, was proposed by Li (1991); see Cook
(1998, Chapter 11).  In \sir, we
make use of the fact that given certain assumptions on the marginal
distribution of $X$\footnote{For the methods discussed here, we generally
need the linearity condition, the $\E(A'x|B'x)$ is a linear function for
$B'x$,
where $A$ is any matrix, and $B$ is a matrix such that $F(y|x) = F(y|B'x)$.
This condition holds, for example, if the marginal distribution of $X$ is
normal, although that assumption is much stronger than is needed.  See Cook
(1998), Cook and Weisberg (1994, 1999) for more discussion.}, the inverse regression problem $F(z|y) \subseteq
\span(B)$.  The general computational outline for \sir\ is as follows:
\begin{enumerate}
    \item Examine $F(z|y)$ by dividing the range of $Y$ into $h$ slices,
    each with approximately the same number of observations.  With a
    multivariate response ($Y$ has $k$ columns), divide the range of
    $Y_1 \times Y_2 \ldots \times Y_k$ into $h$ cells.  For example,
    when
    $k = 3$, and we slice $Y_1$ into 3 slices, $Y_2$ into 2 slices, and
    $Y_3$ into 4 slices, we will have $h=2\times3\times4=24$ cells.
    The number
    of slices or cells $h$ is a tuning parameter of the procedure.
    \item Assume that within each slice or cell $F(z|y)$ is approximately
    constant.  Then the expected value of the within-slice vector of
    sample means will be a vector in $\span(B)$.
    \item  Form the $h \times p$ matrix whose $i$-th row is
    the vector of weighted sample means in the $i$-th slice.
    The matrix $\hat{M}$
    is the $p \times p$ sample covariance matrix of these sample mean
    vectors.
\end{enumerate}
\sir\ thus concentrates on the mean function $\E(Z|Y)$, and ignores any
other dependence.

The output given in the last section is an example of typical output for
\sir.  First is given the eigenvectors and eigenvalues of $\hat{M}$; the
eigenvectors have been back-transformed to the original $X$-scale.
Assuming that the dimension is $d$, the estimate of $\span(B)$ is given by
the first $d$ eigenvectors.  Also given along with the eigenvectors is the
square of the correlation between the ols fitted values and the first $d$
principal directions.  The first direction selected by \sir\ is almost
always about the same as the first direction selected by ols, as is the case
in the example above.

For \sir, Li (1991) provided asymptotic tests of dimension based on partial
sums of eigenvalues, and these tests are given in the summary.  The tests
have asymptotic Chi-square distributions, with the number of degrees of
freedom shown in the output.

Examining the tests shown in the final output, we see that the
test of $d=0$ versus $d>0$ has a very small $p$-value, so we would
reject $d=0$.  The test for $d=1$ versus $d>1$ has $p$-value near
$0.001$, suggesting that $d$ is at least 2.  The test for $d=2$
versus $d>2$ has $p$-value of about $0.31$, so we suspect that
$d=2$ for this problem.  This suggests that further analysis of
this regression problem can be done based on the 3D graph of the
response versus the linear combinations of the predictors
determined by the first two eigenvectors, and the dimension of the
problem can be reduced from 4 to 2 without loss of information.
See Cook (1998), and Cook and Weisberg (1994, 1999), for further
examples and interpretation.

When the response is multivariate, the format of the call is:
\small\begin{verbatim}
m1 <- dr(cbind(LBM,RCC)~Ht+Wt+WCC))
\end{verbatim}
\normalsize
The summary for a multivariate response is similar:
\small \begin{verbatim}
> summary(m1)

Call:
dr(formula = cbind(LBM, RCC) ~ Ht + Wt + WCC)

Terms:
cbind(LBM, RCC) ~ Ht + Wt + WCC

Method:
sir with 9 slices, n = 202, using weights.

Slice Sizes:
24 23 23 23 22 21 22 22 22

Eigenvectors:
      Dir1    Dir2    Dir3
Ht  0.4857  0.3879  0.1946
Wt  0.8171 -0.2238 -0.1449
WCC 0.3105 -0.8941  0.9701

              Dir1    Dir2    Dir3
Eigenvalues 0.7076 0.05105 0.02168
R^2(LBM|dr) 0.9911 0.99124 1.00000
R^2(RCC|dr) 0.9670 0.97957 1.00000

Asymp. Chi-square tests for dimension:
              Stat df p-value
0D vs >= 1D 157.63 24  0.0000
1D vs >= 2D  14.69 14  0.3995
2D vs >= 3D   4.38  6  0.6254
\end{verbatim}
\normalsize
The test statistics are the same as in the univariate response case, as is
the interpretation of the eigenvalues and vectors.  The output gives the
squared correlation of each of the responses with the eigenvectors.

\subsection{Sliced average variance estimation}
Sliced average variance estimation, or \save, was proposed by Cook and
Weisberg (1991).  As with \sir, we slice the range of $Y$ into $h$ slices,
but rather than compute the within-slice mean we compute within-slice
covariance matrices.  If $C_i$ is the weighted within slice sample covariance
matrix in slice $i$,
then the matrix $\hat{M}$ is given by
\[
\hat{M} = \frac{1}{n}\sum g_i(I-C_i)^2
\]
where $g_i$ is the sum of the weights in the slice; if all weights are
equal, then the $g_i$ are just the number of observations in each slice.
\save\ looks at
second moment information and may miss first-moment information,
particularly it may miss linear trends.

Output for \save\ is similar to \sir, except that no asymptotic tests have
been developed.  However, tests of dimension based on a permutation test are
available; see Section~\ref{sec:perm}.

\subsection{Principal Hessian direction}
Li (1992) proposed the method called principal Hessian directions, or pHd.
This method examines the matrix $\hat{M}$ given by
\[
\hat{M} = \frac{1}{n}\sum_{i=1}^n w_i f_i z_iz_i'
\]
where $f_i$ is either equal to $y_i$ (method \phdy) or $f_i$ is an ols
residual (method \phd\ or \phdres), and $w_i$ is the weight for the $i$-th
observation (recall again that we assume that $\sum w_i = n$, and if this is
not satisfied the program rescales the weights to meet this condition).
%If the method is \phdq, then $f_i$ is the $i$th residual from the full
%quadratic ols fit of $y$ to the predictors.
While all methods produce $\hat{M}$
matrices whose eigenvectors are consistent estimates
of vectors in $\span(B)$, the
residual methods are more suitable for tests of dimension.  See Cook (1998,
Chapter 12) for details.

Output for \phd\ is again similar to \sir, except for the tests.  Here is
the output for the same setup as before, but for method \phdres:
\small
\begin{verbatim}
> i2 <- update(i1,method="phdres")
> summary(i2)

Call:
dr(formula = LBM ~ Ht + Wt + log(RCC) + WCC, method = "phdres")

Terms:
LBM ~ Ht + Wt + log(RCC) + WCC

Method:
phd, n = 202.

Eigenvectors:
             Dir1       Dir2      Dir3     Dir4
Ht        0.12764 -0.0003378  0.005550  0.02549
Wt       -0.02163  0.0326138 -0.007342 -0.01343
log(RCC) -0.74348  0.9816463  0.999930 -0.99909
WCC       0.65611 -0.1879008 -0.007408 -0.03157

              Dir1   Dir2   Dir3   Dir4
Eigenvalues 1.4303 1.1750 1.1244 0.3999
R^2(OLS|dr) 0.2781 0.9642 0.9642 1.0000

Asymp. Chi-square tests for dimension:
              Stat df Normal theory Indep. test General theory
0D vs >= 1D 35.015 10     0.0001241    0.005427        0.01811
1D vs >= 2D 20.248  6     0.0025012          NA        0.03200
2D vs >= 3D 10.281  3     0.0163211          NA        0.05530
3D vs >= 4D  1.155  1     0.2825955          NA        0.26625
\end{verbatim}
\normalsize
The column of tests called ``normal theory" were proposed by Li (1992) and
require that the predictors are
normally distributed.  These statistics are asymptotically distributed as
Chi-square, with the degrees of freedom shown.

When the method is \phdres\, additional tests are provided.
Since this method is based on residuals, it gives tests concerning the
central subspace for the regression of the residuals on $X$ rather than the
response on $X$.  The subspace for this residual regression may be, but
need not be, smaller than the subspace for the original regression.
For example, the column
marked ``Indep. test" is essentially a test of $d=0$ versus $d>0$ described
by Cook (1998) for the residual regression.  Should the significance level
for this test be large, we might conclude that the residual regression
subspace is of dimension zero.  From this we have two possible conclusions:
(1) the dimension of the response regression may be 1 if using the residuals
removed a linear trend, or (2) the dimension may be 0 if the residuals did
not remove a linear trend.

Similarly, if the significance level for the independence test is small, then
we can conclude that the dimension is at least 1.  It could be one if the
method is picking up a nonlinear trend in the OLS direction, but it will be
2 if the nonlinearity is in some other direction.

The independence test and the final column, also from Cook
(1998), use the same test statistic, but different distributions based on
different assumptions.
Significance levels are obtained by comparing the statistic to the
distribution of a random linear combination of Chi-square statistics, each
with one df.  These statistics do not require normality
of the predictors.  The way the significance levels
in this column are approximated using the method of Wood (1989).

\subsection{Quadratic \phd}
Li (1992) proposed an alternative method of estimating the Hessian matrix
based on quadratic regression, as follows:  (1) fit the ols regression of
the response on the full second-order model based on all $p$ predictors; (2)
set $\hat{M}$ to be the $p \times p$ matrix whose $(i,j)$ element is the
estimated coefficient of $x_ix_j$.  Given this estimate of $M$, proceed as
with other \phd\ methods.

\subsection{Multivariate \phd}
No multivariate extensions of \phd\ have yet been proposed, and so the
response must be univariate for any \phd\ method.

\subsection{Adding other methods to \ir}\label{sec:newmeth}
You may skip this section unless you are interested in adding
additional dimension reduction methods to \ir. The
\ir\ function is designed to be flexible and to make adding
additional methods to the package as easy as possible. Here are
the steps you need to follow:
\begin{enumerate}
\item Select a name for your method.  For example suppose you
select the name ``poly'' to mean that you will be estimating $M$
by fitting polynomials to the inverse plots each of the predictors
versus the response.  When you call \ir\ with
\dcode{method="poly"}, an object of class \dcode{poly} will be
created.  If your response is a matrix, the object will be of type
\dcode{mpoly}, which inherits from \dcode{poly}.

\begin{table}
\hrule
\caption{The \dcode{dr.fit.M} method for \sir.}
\label{tab:fitM}
\small\begin{verbatim}
#####################################################################
#     Sliced Inverse Regression
#####################################################################

dr.fit.M.sir <-function(object,z,y,w=NULL,nslices=NULL,
                        slice.info=NULL,...) {
# get slice information
    h <- if (!is.null(nslices)) nslices else max(8, NCOL(z)+3)
    slices<- if(is.null(slice.info)) dr.slices(y,h) else slice.info
# initialize slice means matrix
    zmeans <- matrix(0,slices$nslices,NCOL(z))
    slice.weight <- slices$nslices
# make sure weights add to n
    wts <- if(is.null(w)) rep(1,NROW(z)) else NROW(z) * w /sum(w)
# compute weighted means within slice (weights always add to n)
    wmean <- function (x, wts) { sum(x * wts) / sum (wts) }
    for (j in 1:slices$nslices){
      sel <- slices$slice.indicator==j
      zmeans[j,]<- apply(z[sel,],2,wmean,wts[sel])
      slice.weight[j]<-sum(wts[sel])}
# get M matrix for sir
    M <- t(zmeans) %*% apply(zmeans,2,"*",slice.weight)/ sum(slice.weight)
    return (list (M=M,slice.info=slices))
}
\end{verbatim}
\hrule
\normalsize
\end{table}
\item
You will need to write a function called \dcode{dr.fit.M.poly}
that estimates the $M$ matrix for your method.  You can model this
function after the function \dcode{dr.fit.M.sir} shown in
Table~\ref{tab:fitM}.  The important arguments that are passed to
this function include \dcode{z}, which is an $n \times p$ rotated
and centered data matrix with no missing values; \dcode{y}, which
is the response vector or matrix, and \dcode{w}, which is the
vector of weights, if any.  If your method requires other
parameters, for example setting a degree of a polynomial, simply
add the argument
\dcode{degree=2} to the list of function arguments.  This sets the
default value of \dcode{degree} equal to 2.  The ``\ldots''
argument in the functions allow you to add the \dcode{degree}
argument when you call the function \ir.  Your function must
return a list, including the argument \dcode{M}, which is the
matrix of interest. $M$ can be either a square matrix leading to
an analysis of eigenvalues and eigenvectors, as in the example for
\sir, or it can be a rectangular matrix, leading to use of
singular values and vectors.  All entries in the list will become
attributes of the resulting object.  For example, if your list is
\dcode{list(z=z,degree=degree)}, when you create an object like
\small
\begin{verbatim}
> i1 <- dr(LBM~Ht+Wt+RCC+WCC,method="poly",degree=3)
\end{verbatim}
\normalsize
the value of \dcode{i1\$degree} will be 3.
\item
If your method works differently when \dcode{y} is a matrix,
write a method called \dcode{dr.fit.M.mpoly} to do the
computations for this case.  For \sir, the only difference between
univariate and multivariate responses is in the way the slices are obtained,
and the method \dcode{dr.slices} works for either case.  As a result, a
separate multivariate fit method is not required for \sir.
\begin{table}
\hrule
\caption{The \dcode{dr.test.sir} function.}\label{tab:testsir}
\small\begin{verbatim}
dr.test.sir<-function(object,nd) {
#compute the sir test statistic for the first nd directions
    e<-sort(object$evalues)
    p<-length(object$evalues)
    n<-object$cases
    st<-df<-pv<-0
    nt <- min(p,nd)
    for (i in 0:nt-1)
      {st[i+1]<-n*(p-i)*mean(e[seq(1,p-i)])
       df[i+1]<-(p-i)*(object$slice.info$nslices-i-1)
       pv[i+1]<-1-pchisq(st[i+1],df[i+1])
      }
    z<-data.frame(cbind(st,df,pv))
    rr<-paste(0:(nt-1),"D vs >= ",1:nt,"D",sep="")
    dimnames(z)<-list(rr,c("Stat","df","p-value"))
    z
}
\end{verbatim}
\normalsize\hrule
\end{table}

\item
If your method has tests, other than the permutation tests
available for all methods, you will need to write a function
called \dcode{dr.test.poly} (or \dcode{dr.test.mpoly} if a
separate method is required for multivariate responses).  The
equivalent method for \sir\ is shown in Table~\ref{tab:testsir}.
The test method is called by the \dcode{summary.dr} function.
\end{enumerate}

The function \dcode{dr.fit.y(object)} returns the response
variable for use in computing $M$.  For example, the function
\dcode{dr.fit.y.phdy} returns the left-hand side variable from the
formula specified in the call to \dr, while
\dcode{dr.fit.y.phdres} %and \dcode{dr.fit.y.phdq} return, respectively,
returns
the residuals from the regression of the response on the predictors.
%and the residuals from the full quadratic regression of the response on the
%predictors.
You may also want to write a
\dcode{summary.dr.poly} method if the default summary is not
adequate for your needs.

\section{Permutation tests}\label{sec:perm}
Cook (1998) and Yin (2000) discuss permutation tests of dimension that can
be used with a dimension reduction method.  These are implemented in the
function \dcode{dr.permutation.test}.  Typical use of this function is
\small\begin{verbatim}
> dr.permutation.test(i1,npermute=499)

Permutation tests
Number of permutations:
[1] 499

Test results:
               Stat p-value
0D vs >= 1D 219.205   0.000
1D vs >= 2D  41.870   0.002
2D vs >= 3D  11.534   0.284
3D vs >= 4D   3.509   0.354
\end{verbatim}
\normalsize
\noindent
The function requires the name of the object.  The number of permutations
defaults to 50 and the number of directions defaults to 3.  Increasing
either can increase the computation time required to obtain the solution.
The permutation test results for the example
are very similar to the asymptotic results given earlier.

\section{Graphical methods}
The function call \dcode{plot(i1)} returns a scatterplot matrix of the
response and the principal directions.  The call
\dcode{plot(i1,mark.by.y=TRUE)} produces a scatterplot matrix of the
principal directions, but with points marked (colored) according to the
value of the response; the point marking does not work with \Splus.

The function call \dcode{dr.coplot(i1)} returns a plot of the response versus
the first principal direction, with a separate panel conditioning on the
value of the second principal direction (a coplot).  The call
\dcode{dr.coplot(i1,mark.by.y=T)}, gives a coplot of the first direction
versus the second direction conditioning on the third direction and using
color to encode the information about the response.

The function \dcode{rotplot} is a generic function that allows looking at a
number of static views of a 3D plot.  The call
\small\begin{verbatim}
> rotplot(dr.directions(m1,1:2),dr.y(m1),number=16)
\end{verbatim}
\normalsize
\noindent
gives 16 views of the 3D plot of the response versus linear combinations of
the first two principal directions.

\section{Weights}
Weights are generally used in dimension reduction methods to make the resulting
weighted sample closer to a normal distribution than the original sample.
Cook (1998, Section 8.4) discusses the method that is implemented here.
When weights are present, they are used in centering the data and computing
the covariance matrix, and they are used in computing the objective matrix
$M$ for \phd.
%For consistency with \arc, they are not used in computing $M$
%for \sir\ and \save\ unless the argument \dcode{use.weights} is set to
%\dcode{T}.
Weights may be provided by the user with the \dcode{weights}
argument.  If
\dcode{weights=NULL}, the default, no weighting is used.

The function \dcode{dr.weights} is used to estimate weights using the
algorithm described by Cook (1998, Sec. 8.4).  There are several
other arguments that control how the weights are computed, as
described below, and on the help page for the function
\dcode{dr.weights}.  The algorithm works as follows:
\begin{enumerate}
    \item For an $n\times p$ data matrix $X$, find estimates $m$ and $S$
    of the mean and covariance matrix.  For this purpose, in \R\ the function
    \dcode{cov.rob} in the \dcode{lqs} package is used, while in \Splus\ the
    function \dcode{covRob} in the \dcode{robust} package is used; in either
    case the needed package will be loaded automatically.
    If you do not want to use one of these routines, you must rewrite the
    function
    \dcode{robust.center.scale} to use your preferred code.
    In \R, the method of computing $m$ and $S$ is determined
    by the argument \dcode{covmethod}.  If \dcode{covmethod="classical"}, the
    usual estimator is used for $S$, but $m$ is estimated by medians.  If
    \dcode{method="mve"}, the default, or \dcode{method="mcd"}, the
    covariance matrix is estimated by the minimum volume ellipsoid method
    and the minimum determinant method, respectively.  These latter two also
    return a robust estimate of center.  Any tuning parameters for
    the method to compute the robust estimate of $m$ and $S$
    can be passed from the call to \dr.
    See the documentation for
    \dcode{cov.rob} for a description of these additional parameters.  All
    the defaults are sensible, so most users will not need to use these
    additional parameters.
   \item Compute the matrix $Z = (X - 1m')S^{-1/2}$.  If the data were
   normally distributed $\N(m,S)$, the rows of $Z$ would be like a
   sample from $\N(0,I)$.
   \item Obtain a random vector $b$ from the $\N(0,\sigma^2I)$ distribution.
   The parameter \dcode{sigma=1} is a tuning parameter that can be set in
   the call to \ir, and values near 1 or slightly smaller seem
   appropriate.  Find the row of $Z$ that is closest to $b$ (the code uses
   Euclidean distance), and increase a counter for that row by 1.
   \item The argument \dcode{nsamples} determines the number of times this
   last step is repeated; the default is
   \dcode{nsamples=10*dim(x)[1]}
   where $X$ is the $n\times p$ data matrix; this number may be too small.
   \item Return a vector of weights given by the value of the counter
   divided by \dcode{nsamples} and multiplied by $n$, so the sum of the
   weights will be $n$.
   \end{enumerate}

An example of the use of weights is:
\begin{verbatim}
> wts <- dr.weights(LBM~Ht+Wt+RCC+WCC)
> i1 <- dr(LBM~Ht+Wt+RCC+WCC,weights=wts,method="phdres")
\end{verbatim}

\section{Miscellaneous functions included in \ir}
The function \dcode{dr.direction} takes two arguments, the object
name, and which directions are wanted (for, example, \dcode{1:3}
returns the first three directions).  It returns the matrix $XU$,
scaled to have unit column length unless the argument norm is
false, where $U$ gives the specified eigenvectors.

The function calls \dcode{dr.x(i1}) and \dcode{dr.y(i1)} return the model
matrix and the response, respectively.
\dcode{dr.z(i1,center=T,rotate=T)} returns the centered and rotated $Z$ matrix
from $X$.  You can also use \dcode{dr.z} by explicity providing a
matrix $X$ in place of the first argument, and if necessary a
vector of weights as a second argument.

The routine used for slicing is called
\dcode{dr.slices(y,h)} to slice \dcode{y} into \dcode{h} slices.  If \dcode{y} has
$p$ columns and \dcode{h} has $p$ elements, then slicing is done
recursively.  The first column of \dcode{y} is sliced into
\dcode{h[1]} slices.  Within each of these slices, the second
column of \dcode{y} is sliced into \dcode{h[2]} slices, giving
\dcode{h[1]*h[2]} slices. This process is then repeated for any
additional columns.  If \dcode{h} is a scalar, then each dimension
is sliced into the smallest integer larger than \dcode{h}$^{1/p}$
slices.  For example, if $p=2$ and \dcode{h}$=8$, then each
dimension has 3 slices for a total of 9.


\section{Bug Reports}
Please send bug reports to
\htmladdnormallink{sandy@stat.umn.edu}{mailto:sandy@stat.umn.edu}.

\section{\Splus}
\dr\ has been tested with \Splus\ versions 5 and 6
under Linux.  If you want to use \dr\ with \Splus, you must change the
value of \dcode{whichengine} on line 34 of the file R/dr to
\dcode{whichengine <- "s6"} before you load the file.
All options available in \R\ seem to work in \Splus, except that the argument
\dcode{mark.by.y} on plots and coplots and the function
\dcode{markby} do not work with \Splus.

\section{Acknowledgements}
Jorge de la Vega did extensive testing of the code, and wrote some of the
functions, and Cindy Yu wrote early versions of some of the functions.
Referee's comments were very helpful in revising this paper and in
documenting \dr.  Kurt Hornik was very helpful in getting the package to
pass all the tests built-in to \R.


\begin{references}

%\item Bentler, Peter M. and Xie, Jun (2000).  corrections to test statistics
%in principal Hessian directions.  {\em Statistics and Probability Letters},
%47, 381--389.

\item Cook, R. D. (1998). {\it Regression Graphics}.  New York:  Wiley.

\item Cook, R. D. and Weisberg, S. (1991).  Discussion of Li (1991).
\jasa, 86, 328--332.

\item Cook, R. D. and Weisberg, S. (1994).  {\it An Introduction to
Regression Graphics}.  New York:  Wiley.

\item Cook, R. D. and Weisberg, S. (1999). {\it Applied Regression
Including Computing and Graphics},  New York:  Wiley.

\item Li, K C. (1991).  Sliced inverse regression for dimension reduction
(with discussion), \jasa, 86, 316-342.

\item Li, K C. (1992).  On principal Hessian directions for data
visualization and dimension reduction:  Another application of Stein's
lemma.  \jasa, 87, 1025--1034.

\item Wood, A. (1989).  An F-approximation to the distribution of a linear
combination of chi-squared random variables.  {\em Communication in
Statistics, Part B -- Simulation and Computation}, 18, 1439--1456.

\item Yin, Xiangrong (2000).  Dimension reduction using inverse third
moments and central $k$-th moment subspaces.  Unpublished Ph. D.
dissertation, University of Minnesota, School of Statistics.
\end{references}

\newpage
\section{Function documentation}
\small

\Header{dr}{Dimension reduction regression}
\methalias{dr.compute}{dr}
\alias{plot.dr}{dr}
\alias{print.dr}{dr}
\alias{print.summary.dr}{dr}
\alias{summary.dr}{dr}
\keyword{regression}{dr}
\begin{Description}\relax
The function dr implements dimension reduction methods, including SIR, SAVE and pHd.
'dr' calls 'dr.compute', so only the former will be needed by most users.
\end{Description}
\begin{Usage}
\begin{verbatim}
dr (formula, data, subset, na.action = na.fail, weights, 
    ...)
    
dr.compute (x, y, weights, method = "sir", ...)
 \end{verbatim}
\end{Usage}
\begin{Arguments}
\begin{ldescription}
\item[\code{formula}] a symbolic description of the model to be fit. The details of
the model are the same as for lm. Although factors may not be 
appropriate for dr methods, they are permitted.  Full rank models are
recommended, although rank deficient models are permitted.
\item[\code{data}] an optional data frame containing the variables in the model.
By default the variables are taken from the environment from
which `dr' is called.
\item[\code{subset}] an optional vector specifying a subset of observations to be
used in the fitting process.
\item[\code{weights}] an optional vector of weights to be used where appropriate.  In the
context of dimension reduction methods, weights are used to obtain
elliptical symmetry, not constant variance; see \code{\Link{dr.weights}}.
\item[\code{na.action}] a function which indicates what should happen when the data
contain `NA's.  The default is `na.fail,' which will stop calculations.
The option 'na.omit' is also permitted, but it may not work correctly when
weights are used.
\item[\code{x}] The design matrix
\item[\code{y}] The response vector or matrix
\item[\code{method}] This character string specifies the method of fitting.  ``sir"
specifies sliced inverse regression and ``save" specifies sliced
average variance estimation.  ``phdy" uses principal hessian
directions using the response as suggested by Li, and ``phdres" 
uses the LS residuals as suggested by Cook. Other methods may be
added.
\item[\code{...}] For 'dr', arguments passed to 'dr.compute'.  For 'dr.compute', 
arguments required for particular dimension reduction method.
\code{nslices} is the number of slices used by sir and save.
\code{numdir} is the maximum number of directions to compute, with
default equal to 4. other methods may have other defaults.
\end{ldescription}
\end{Arguments}
\begin{Details}\relax
The general regression problem studies \eqn{F(y|x)}{}, the conditional
distribution of a response \eqn{y}{} given a set of predictors \eqn{x}{}.  
This function provides methods for estimating the dimension and central
subspace of a general regression problem.  That is, we want to find a 
\eqn{p \times d}{p by d} matrix \eqn{B}{} such that 
\deqn{F(y|x)=F(y|B'x)}{}  
Both the dimension \eqn{d}{} and the subspace
\eqn{R(B)}{} are unknown.  These methods make few assumptions.  All the methods
available in this function estimate the unknowns by study of the inverse
problem, \eqn{F(x|y)}{}.  In each, a kernel matrix \eqn{M}{} is estimated such
that the column space of \eqn{M}{} should be close to the central subspace.
Eigenanalysis of \eqn{M}{} is then used to estimate the central subspace.
Objects created using this function have appropriate print, summary and plot
methods.

Weights can be used, essentially to specify the relative 
frequency of each case in the data.  Empirical weights that make 
the contours of the weighted sample closer to elliptical can be 
computed using \code{\Link{dr.weights}}.  
This will usually result in zero weight for some 
cases.  The function will set zero estimated weights to missing.

Several functions are provided that require a dr object as input.  
\code{dr.permutation.tests} uses a permutation test to obtain significance levels
for tests of dimension.  \code{dr.coplot} allows visualizing the results using a
coplot of either two selected directions conditioning on a third and using
color to mark the response, or the resonse versus one direction,
conditioning on a second direction.  \code{plot.dr} provides the default plot
method for dr objects, based on a scatterplot matrix.
\end{Details}
\begin{Value}
dr returns an object that inherits from dr (the name of the type is the
value of the \code{method} argument), with attributes:
\begin{ldescription}
\item[\code{x}] The design matrix
\item[\code{y}] The response vector
\item[\code{weights}] The weights used, normalized to add to n.
\item[\code{qr}] QR factorization of x.
\item[\code{cases}] Number of cases used.
\item[\code{call}] The initial call to 'dr'.
\item[\code{M}] A matrix that depends on the method of computing.  The column space
of M should be close to the central subspace.
\item[\code{evalues}] The eigenvalues of M (or squared singular values if M is not
symmetric).
\item[\code{evectors}] The eigenvectors of M (or of M'M if M is not square and
symmetric) ordered according to the eigenvalues.
\item[\code{numdir}] The maximum number of directions to be found.  The output
value of numdir may be smaller than the input value.
\item[\code{slice.info}] output from 'sir.slice', used by sir and save.
\item[\code{method}] the dimension reduction method used.
\end{ldescription}
 

\code{dr.weights} returns a vector of weights estimated weights, scaled to add to
the number of cases.
\end{Value}
\begin{Author}\relax
Sanford Weisberg, 
sandy@stat.umn.edu 

For weights, see
R. D. Cook and C. Nachtsheim (1994), Reweighting to achieve
elliptically contoured predictors in regression.  Journal of the American
Statistical Association, 89, 592--599.
\end{Author}
\begin{References}\relax
The details of these methods are given by R. D. Cook 
(1998).  Regression Graphics.  New York:  Wiley.  Equivalent 
methods are also available in Arc, R. D. Cook and S. Weisberg 
(1999).  Applied Regression Including Computing and Graphics, New 
York:  Wiley, www.stat.umn.edu/arc.
\end{References}
\begin{SeeAlso}\relax
\code{\Link{dr.permutation.test}},\code{\Link{dr.x}},\code{\Link{dr.y}},
\code{\Link{dr.direction}},\code{\Link{dr.coplot}},\code{\Link{dr.weights}}
\end{SeeAlso}
\begin{Examples}
\begin{ExampleCode}
library(dr)
data(ais)
attach(ais)  # the Australian athletes data
#fit dimension reduction using sir
m1 <- dr(LBM~Wt+Ht+RCC+WCC, method="sir", nslices = 8)
summary(m1)

# repeat, using save:

m2 <- update(m1,method="save")
summary(m2)

# repeat, using phd:

m3 <- update(m2, method="phdres")
summary(m3)
\end{ExampleCode}
\end{Examples}



\input{latex/dr.estimate.weights.tex}

\Header{dr.permutation.test}{Dimension Reduction Regression Functions}
\methalias{dr.permutation.test.statistic}{dr.permutation.test}
\alias{dr.permutation.test.statistic.default}{dr.permutation.test}
\alias{dr.permutation.test.statistic.phd}{dr.permutation.test}
\alias{dr.permutation.test.statistic.phdres}{dr.permutation.test}
\alias{dr.permutation.test.statistic.phdy}{dr.permutation.test}
\alias{print.dr.permutation.test}{dr.permutation.test}
\alias{summary.dr.permutation.test}{dr.permutation.test}
\keyword{regression}{dr.permutation.test}
\begin{Description}\relax
These functions require a dimension reduction regression object as input to
produce output of various types.
\end{Description}
\begin{Usage}
\begin{verbatim}
dr.permutation.test(object, npermute=50,numdir=object$numdir,permute.weights=TRUE)

\end{verbatim}
\end{Usage}
\begin{Arguments}
\begin{ldescription}
\item[\code{object}] a dimension reduction regression object created by dr
\item[\code{npermute}] number of permutations to compute, default is 50
\item[\code{numdir}] maximum permitted value of the dimension, with the default from
the object
\item[\code{permute.weights}] If TRUE, permute weights as well as data in the
permutation test.  If FALSE, do not permute the weights.
\end{ldescription}
\end{Arguments}
\begin{Details}\relax
Approximates significance levels for tests of dimension using a permutation test.
\end{Details}
\begin{Value}
Returns an object of type 'dr.permutation.test' that can be printed or
summarized to give the summary of the test.
\end{Value}
\begin{Author}\relax
Sanford Weisberg, sandy@stat.umn.edu
\end{Author}
\begin{References}\relax
See www.stat.umn.edu/arc/addons.html, and then select the article
on dimension reduction regression or inverse regression.
\end{References}
\begin{SeeAlso}\relax
\code{\Link{dr}}
\end{SeeAlso}
\begin{Examples}
\begin{ExampleCode}
data(ais)
attach(ais)  # the Australian athletes data
#fit dimension reduction regression using sir
m1 <- dr(LBM~Wt+Ht+RCC+WCC, method="sir", nslices = 8)
summary(m1)
dr.permutation.test(m1,npermute=100)
plot(m1)
dr.coplot(m1)
\end{ExampleCode}
\end{Examples}



\Header{plot.dr}{Plotting methods for dimension reduction regression}
\alias{coplot.dr}{plot.dr}
\keyword{regression}{plot.dr}
\begin{Description}\relax
These routines provide default plotting methods for dimension reduction regression.\end{Description}
\begin{Usage}
\begin{verbatim}
plot.dr(object, which=1:object$numdir, mark.by.y=F,plot.method=pairs, ...)
dr.coplot(object, which=1:object$numdir, mark.by.y=F, ...)
\end{verbatim}
\end{Usage}
\begin{Arguments}
\begin{ldescription}
\item[\code{object}] Any dimension reduction regression object
\item[\code{which}] Which directions to plot
\item[\code{mark.by.y}] if TRUE, use the response as a marking variable to color
points; if FALSE, use response in the plot
\item[\code{plot.method}] The default is to use the pairs or coplot method to
draw the plots.  If John Fox's car library is available, you can
substitute scatterplot.matrix for pairs.
\item[\code{...}] arguments passed to plot or coplot.  In particular, if the
car library is available, the argument panel=panel.car will add smoothers
to a coplot.
\end{ldescription}
\end{Arguments}
\begin{Value}
Produces a scatterplot matrix (plot) or coplot (dr.coplot) of the specified
directions in an dimension reduction regression\end{Value}
\begin{Author}\relax
Sanford Weisberg, sandy@stat.umn.edu\end{Author}
\begin{References}\relax
Cook, R. D. and Weisberg, S. (1999).  Applied Regression
Including Computing and Graphics.  New York:  Wiley.\end{References}
\begin{Examples}
\begin{ExampleCode}
data(ais)
attach(ais)
i1<-dr(LBM~Ht+Wt+RCC+WCC)
plot(i1)
dr.coplot(i1,mark.by.y=TRUE)
\end{ExampleCode}
\end{Examples}



\Header{rotplot}{draw many 2D projections of a 3D plot}
\keyword{regression}{rotplot}
\begin{Description}\relax
This function draws several 2D views of a 3D plot, sort of like a spinning
plot.
\end{Description}
\begin{Usage}
\begin{verbatim}
rotplot(x, y, number=9, theta=seq(0, pi/2, length = number), ...)
\end{verbatim}
\end{Usage}
\begin{Arguments}
\begin{ldescription}
\item[\code{x}] a matrix with 2 columns giving the horizontal axes of the full 3D
plot.
\item[\code{y}] the vertical axis of the 3D plot.
\item[\code{number}] Number of 2D views of the 3D plot.
\item[\code{theta}] a list of rotation angles
\item[\code{...}] additional arguments passed to coplot
\end{ldescription}
\end{Arguments}
\begin{Details}\relax
For each value of theta, draw the plot of cos(theta)*x[,1]+sin(theta)*x[,2]
versus y.
\end{Details}
\begin{Value}
returns a graph object.
\end{Value}
\begin{Author}\relax
Sanford Weisberg, sandy@stat.umn.edu
\end{Author}
\begin{Examples}
\begin{ExampleCode}
 data(ais)
 attach(ais)
 m1 <- dr(LBM ~ Ht + Wt + WCC)  
 rotplot(dr.direction(m1,which=1:2),dr.y(m1),col=markby(Sex))
  \end{ExampleCode}
\end{Examples}



\Header{dr.x}{Inverse regression term matrix}
\alias{dr.y}{dr.x}
\alias{dr.z}{dr.x}
\keyword{Dimension reduction regression, regression}{dr.x}
\begin{Description}\relax
dr.x returns the matrix of data constructed from the formula for a
dimension reduction regression.  dr.y returns the response.\end{Description}
\begin{Usage}
\begin{verbatim}
dr.x(object)
dr.y(object)
dr.z(x,weights=NULL,center=TRUE,rotate=TRUE,decomp="svd")
\end{verbatim}
\end{Usage}
\begin{Arguments}
\begin{ldescription}
\item[\code{object}] An dimension reduction regression object
\item[\code{x}] 
\item[\code{weights}] 
\item[\code{center}] 
\item[\code{rotate}] 
\item[\code{decomp}] 
\end{ldescription}
 Decomposition to be used in computing the rotation; the
default is "svd".\end{Arguments}
\begin{Value}
dr.x returns an \eqn{n \times p}{n by p} matrix of terms, excluding the 
intercept, constructed from the dimension reduction regression object.  dr.y returns the
response. dr.z returns a possibly centered and scaled version of x.\end{Value}
\begin{Author}\relax
Sanford Weisberg, <sandy@stat.umn.edu>\end{Author}
\begin{SeeAlso}\relax
dr\end{SeeAlso}
\begin{Examples}
\begin{ExampleCode}
data(ais)
attach(ais)
m1 <- dr(LBM~Ht+Wt+RCC+WCC)
dr.x(m1)
dr.y(m1)
\end{ExampleCode}
\end{Examples}




{\it The following documentation is from the package lqs by Brian Ripley,
and is included here for convenience:}

\Header{cov.rob}{Resistant Estimation of Multivariate Location and Scatter}
\alias{cov.mcd}{cov.rob}
\alias{cov.mve}{cov.rob}
\keyword{robust}{cov.rob}
\keyword{multivariate}{cov.rob}
\begin{Description}\relax
Compute a multivariate location and scale estimate with a high
breakdown point -- this can be thought of as estimating the mean and
covariance of the \code{good} part of the data. \code{cov.mve} and
\code{cov.mcd} are compatibility wrappers.\end{Description}
\begin{Usage}
\begin{verbatim}
cov.rob(x, cor = FALSE, quantile.used = floor((n + p + 1)/2),
        method = c("mve", "mcd", "classical"), nsamp = "best", seed)
cov.mve(x, cor = FALSE, quantile.used = floor((n + p + 1)/2),
        nsamp = "best", seed)
cov.mcd(x, cor = FALSE, quantile.used = floor((n + p + 1)/2),
        nsamp = "best", seed)
\end{verbatim}
\end{Usage}
\begin{Arguments}
\begin{ldescription}
\item[\code{x}] a matrix or data frame.

\item[\code{cor}] should the returned result include a correlation matrix?

\item[\code{quantile.used}] the minimum number of the data points regarded as \code{good} points.

\item[\code{method}] the method to be used -- minimum volume ellipsoid, minimum
covariance determinant or classical product-moment. Using
\code{cov.mve} or \code{cov.mcd} forces \code{mve} or \code{mcd}
respectively.

\item[\code{nsamp}] the number of samples or \code{"best"} or \code{"exact"} or
\code{"sample"}.
If \code{"sample"} the number chosen is \code{min(5*p, 3000)}, taken
from Rousseeuw and Hubert (1997). If \code{"best"} exhaustive
enumeration is done up to 5000 samples: if \code{"exact"}
exhaustive enumeration will be attempted however many samples are needed.

\item[\code{seed}] the seed to be used for random sampling: see \code{\Link{RNGkind}}. The
current value of \code{.Random.seed} will be preserved if it is set.

\end{ldescription}
\end{Arguments}
\begin{Details}\relax
For method \code{"mve"}, an approximate search is made of a subset of
size \code{quantile.used} with an enclosing ellipsoid of smallest volume; in
method \code{"mcd"} it is the volume of the Gaussian confidence
ellipsoid, equivalently the determinant of the classical covariance
matrix, that is minimized. The mean of the subset provides a first
estimate of the location, and the rescaled covariance matrix a first
estimate of scatter. The Mahalanobis distances of all the points from
the location estimate for this covariance matrix are calculated, and
those points within the 97.5\% point under Gaussian assumptions are
declared to be \code{good}. The final estimates are the mean and rescaled
covariance of the \code{good} points.

The rescaling is by the appropriate percentile under Gaussian data; in
addition the first covariance matrix has an \emph{ad hoc} finite-sample
correction given by Marazzi.

For method \code{"mve"} the search is made over ellipsoids determined
by the covariance matrix of \code{p} of the data points. For method
\code{"mcd"} an additional improvement step suggested by Rousseeuw and
van Driessen (1997) is used, in which once a subset of size
\code{quantile.used} is selected, an ellipsoid based on its covariance
is tested (as this will have no larger a determinant, and may be smaller).\end{Details}
\begin{Value}
A list with components

\begin{ldescription}
\item[\code{center}] the final estimate of location.

\item[\code{cov}] the final estimate of scatter.

\item[\code{cor}] (only is \code{cor = TRUE}) the estimate of the correlation
matrix.

\item[\code{sing}] message giving number of singular samples out of total

\item[\code{crit}] the value of the criterion on log scale. For MCD this is
the determinant, and for MVE it is proportional to the volume.

\item[\code{best}] the subset used. For MVE the best sample, for MCD the best
set of size \code{quantile.used}.

\item[\code{n.obs}] total number of observations.

\end{ldescription}
\end{Value}
\begin{Author}\relax
B.D. Ripley\end{Author}
\begin{References}\relax
P. J. Rousseeuw and A. M. Leroy (1987) 
\emph{Robust Regression and Outlier Detection.}
Wiley.

A. Marazzi (1993) 
\emph{Algorithms, Routines and S Functions for Robust Statistics.}
Wadsworth and Brooks/Cole. 

P. J. Rousseeuw and B. C. van Zomeren (1990) Unmasking
multivariate outliers and leverage points, 
\emph{Journal of the American Statistical Association}, \bold{85}, 633--639.

P. J. Rousseeuw and K. van Driessen (1999) A fast algorithm for the
minimum covariance determinant estimator. \emph{Technometrics}
\bold{41}, 212--223.

P. Rousseeuw and M. Hubert (1997) Recent developments in PROGRESS. In
\emph{L1-Statistical Procedures and Related Topics }
ed Y. Dodge, IMS Lecture Notes volume \bold{31}, pp. 201--214.\end{References}
\begin{SeeAlso}\relax
\code{\Link{lqs}}\end{SeeAlso}
\begin{Examples}
\begin{ExampleCode}
data(stackloss)
set.seed(123)
cov.rob(stackloss)
cov.rob(stack.x, method = "mcd", nsamp = "exact")
\end{ExampleCode}
\end{Examples}



\end{document}
